% Vim: set spelllang=fr
% PAQUETS
%\usepackage{amsfonts}
\usepackage{amsmath}
%\usepackage{amssymb}
%\usepackage{amsthm}
\usepackage{array}
\usepackage[backend=biber,language=french]{biblatex}
\usepackage{booktabs}
\usepackage{color}
%\usepackage{comment}
\usepackage{csquotes}
%\usepackage{fancybox}
%\usepackage{fancyhdr}
%\usepackage{float}
%\usepackage{flushend}
%\usepackage[left=2cm, top=2cm, right=2cm, bottom=2cm]{geometry}
\usepackage{graphicx}
\usepackage[unicode]{hyperref}
\usepackage{iftex}
%\usepackage{listings}
%\usepackage{multicol}
%\usepackage{multirow}
%\usepackage{setspace}
\usepackage{xcolor}
\usepackage{xspace}
\usepackage{xpatch}
\usepackage{wrapfig}

% FONTES & LANGAGES
\ifXeTeX\else
\ifLuaTeX\else
\begingroup
  \errorcontextlines=-1\relax
  \newlinechar=10\relax
  \errmessage{^^J
  *************************************************^^J
  * (Lua|Xe)TeX is required to compile this document.^^J
  * Sorry!^^J
  *************************************************}%
\endgroup
\fi\fi

\ifLuaTeX
    \usepackage{luatextra}
    \usepackage{microtype}
\else
    \usepackage{fontspec}
\fi
\usepackage{polyglossia}

\setdefaultlanguage{french}
\setotherlanguage{english}
\defaultfontfeatures{Ligatures=TeX}
\setmainfont[%
        UprightFeatures={StylisticSet=2},%
        BoldFeatures={StylisticSet=1},%
        ItalicFeatures={StylisticSet=1},%
        %BoldItalicFeatures={StylisticSet=1},%
    ]{Linux Libertine O}
    \setsansfont[%
        UprightFeatures={StylisticSet=2},%
        BoldFeatures={StylisticSet=1},%
        ItalicFeatures={StylisticSet=1},%
        BoldItalicFeatures={StylisticSet=1},%
    ]{Linux Biolinum O}
\setmonofont{Inconsolata}

% HYPERSETUP
\hypersetup{
    linktoc=all,
    %backref=true,
    breaklinks,
    colorlinks,
    linkcolor=black,
    citecolor=black,
    urlcolor=blue,
    baseurl       = http://,
    pdfpagelayout = OneColumn, % pdfpagelayout=SinglePage
    pdfstartpage  = 1,
    pdfcreator    = {Vim, \LaTeX{}, and a lot of packages!},
    %pdfproducer   = {\LaTeX{}},% Contient automatiquement le compilateur
    bookmarksopen = true,
    bookmarksdepth= 2,
    pdfauthor     = {Quentin MONNET},
    pdftitle      = {Manuscrit de thèse},
    pdfsubject    = {Réseaux de capteurs et dénis de service},
    pdfkeywords   = {Thèse~; Manuscrit}
}

% BIBLIO
\addbibresource{Back/biblio.bib}

% OPTIONS
\renewcommand{\labelitemi}{\textbullet}
