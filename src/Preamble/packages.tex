% Vim: set spelllang=fr:
% PAQUETS
%\usepackage{amsfonts}
\usepackage{amsmath}
\usepackage{amssymb}
\usepackage{amsthm}
\usepackage{array}
\usepackage[backend=biber,language=french,backref=true,defernumbers=true,maxnames=12]{biblatex}
\usepackage{booktabs}
\usepackage{color}
%\usepackage{comment}
\usepackage{csquotes}
%\usepackage{fancybox}
%\usepackage{fancyhdr}
\usepackage{float}
%\usepackage{flushend}
%\usepackage[left=2cm, top=2cm, right=2cm, bottom=2cm]{geometry}
\usepackage{graphicx}
\usepackage[unicode]{hyperref}
\usepackage{iftex}
\usepackage{lettrine}
%\usepackage{listings}
\usepackage{metalogo}
%\usepackage{multicol}
\usepackage{multirow}
\usepackage{setspace}
\usepackage{tikz}
\usepackage{xcolor}
\usepackage{xspace}
\usepackage{xpatch}
\usepackage{wrapfig}

% Chap. théorie des jeux
\usetikzlibrary{automata,calc,shapes,mindmap,shadows}
\theoremstyle{definition}
\newtheorem{theorem}{Théorème}                                                    %
\newtheorem{lemma}[theorem]{Lemme}
\newtheorem{definition}{Définition}
\newtheorem{remark}{Remarque}

% FONTES & LANGAGES
\ifXeTeX\else
\ifLuaTeX\else
\begingroup
  \errorcontextlines=-1\relax
  \newlinechar=10\relax
  \errmessage{^^J
  *************************************************^^J
  * (Lua|Xe)TeX is required to compile this document.^^J
  * Sorry!^^J
  *************************************************}%
\endgroup
\fi\fi

\ifLuaTeX
    \usepackage{luatextra}
    \usepackage{microtype}
\else
    \usepackage{fontspec}
\fi
\usepackage{polyglossia}

\setdefaultlanguage{french}
\setotherlanguage{english}
\defaultfontfeatures{Ligatures=TeX}
%\defaultfontfeatures{Ligatures={NoRequired,NoCommon,NoContextual}}
\setmainfont[%
        UprightFeatures={StylisticSet=2},%
        BoldFeatures={StylisticSet=1},%
        ItalicFeatures={StylisticSet=1},%
        %BoldItalicFeatures={StylisticSet=1},%
    ]{Linux Libertine O}
\setsansfont[%
        UprightFeatures={StylisticSet=2},%
        BoldFeatures={StylisticSet=1},%
        ItalicFeatures={StylisticSet=1},%
        BoldItalicFeatures={StylisticSet=1},%
    ]{Linux Biolinum O}
\setmonofont{Inconsolata}

\makeatletter
    \definecolor{blueLACL}{RGB}{\@blueLACL}
    \definecolor{chapterLACL}{RGB}{\@chapterLACLbis}
\makeatother

% HYPERSETUP
\hypersetup{%
    linktoc=all,
    %backref=true,
    breaklinks,
    colorlinks,
    linkcolor     = blueLACL,
    citecolor     = blueLACL,
    urlcolor      = blueLACL,
    baseurl       = http://,
    pdfpagelayout = OneColumn, % pdfpagelayout = SinglePage
    pdfstartpage  = 1,
    pdfcreator    = {Vim, waf, gnuplot, dia, Inkscape, LaTeX (XeLaTeX), et tout un tas de paquets !},
    %pdfproducer   = {\LaTeX{}},% Contient automatiquement le compilateur
    bookmarksopen = true,
    bookmarksdepth= 2,
    pdfauthor     = {Quentin MONNET},
    pdftitle      = {Modèles et mécanismes pour la protection contre les attaques par déni de service dans les réseaux de capteurs sans fil},
    pdfsubject    = {Thése de doctorat en informatique, Université Paris-Est},
    pdfkeywords   = {Réseaux de capteurs sans fil~; Énergie~; Sécurité~; Déni de service~; Détection d'intrusion~; Simulation~; Modélisation~; Théorie des jeux}
}

% NUMEROS LIGNES
%\usepackage[modulo]{lineno}
%\linenumbers

% MINITOC
\usepackage{minitoc}
\mtcsetdepth{minitoc}{1}
\mtcsettitle{minitoc}{}
%\mtcselectlanguage{french}

% BIBLIO
\addbibresource{Back/biblio.bib}
\addbibresource{Back/thjeux.bib}
% \chapter instead of \chapter*
\defbibheading{bibliography}[\bibname]{\chapter{#1}}%\markboth{#1}{#1}}
%\DeclareSourcemap{%
    %\maps[datatype=bibtex]{%
        %\map{%
            %\step[fieldsource=author, match={Quentin Monnet}, final]%
            %\step[fieldset=keywords, fieldvalue=qmo]%
        %}%
    %}%
%}

% INDEX
\usepackage{makeidx}
\makeindex
\usepackage[intoc]{nomencl}
\renewcommand{\nomname}{Liste des sigles}
\makenomenclature

% OPTIONS
\renewcommand{\labelitemi}{\textbullet}

% TITRES SECTIONS, SOUS-SECTIONS
\usepackage{titlesec}
\colorlet{SecHead}{blueLACL}
\colorlet{SecTab}{chapterLACL!50}
\colorlet{SecText}{white}
\colorlet{SSHead}{blueLACL}
\colorlet{SSText}{white}
%\colorlet{SecHead}{chapterLACL!20}
%\colorlet{SecTab}{chapterLACL}
%\colorlet{SecText}{black}
%\colorlet{SSHead}{chapterLACL}
%\colorlet{SSText}{white}
\newlength\sectionnode
\setlength\sectionnode{\textwidth}
\addtolength\sectionnode{-16.4pt} % 2 * (inner sep + border)
\newcommand{\sectionheader}[1]{%
  \begin{tikzpicture}
  \node[draw,color=SecHead,fill=SecHead,inner sep=8pt] (sectiontitle) {\begin{minipage}{\sectionnode}\color{white}\color{SecText}\thesection \quad #1\end{minipage}};
    \fill[fill=SecTab] (sectiontitle.north west) -- ++(0,8pt) -- ++(3cm,0) -- +(0.3cm,-8pt) -- cycle;
  \end{tikzpicture}%
}
\titleformat{\section}
  [hang]% style : hang, display, runin, leftmargin, ...
  {\Large}% fonte numéro + titre
  {}% numéro
  {0em}% espace entre le numéro et le titre
  {\sectionheader}% fonte titre
%\titlespacing*{\section}
  %{0pt}% retrait à gauche
  %{2em plus 0.3em minus .1em}% espace avant
  %{1.5em plus 0.25em}% espace après
  %[0pt]% retrait à droite

\newcommand{\newthesubsection}{%
  \begin{tikzpicture}[baseline=(noexo.base)]
    \node[right,inner sep=3pt] (noexo) {\phantom{\thesubsection}};
    \filldraw[color=SSHead, rounded corners=1pt] (noexo.south west) -- (noexo.north west) -- ($ (noexo.north east) + (6pt,0)$) -- ($ (noexo.south east) + (3pt,0)$) -- cycle;
    \node[right,inner sep=3pt] {\color{SSText}\thesubsection};
  \end{tikzpicture}%
  \vrule height 0em depth 0.75em width 0pt%
}
\titleformat{\subsection}
  [hang]% style : hang, display, runin, leftmargin, ...
  {\normalfont\large\bfseries}% fonte numéro + titre
  {\newthesubsection}% numéro
  {0.5em}% espace entre le numéro et le titre
  {}% fonte titre
%\titlespacing*{\subsection}
  %{0pt}% retrait à gauche
  %{1em plus 0.3em minus .1em}% espace avant
  %{0.5em plus 0.25em}% espace après
  %[0pt]% retrait à droite

\titleformat{\subsubsection}
  [hang]% style : hang, display, runin, leftmargin, ...
  {\normalfont\bfseries}% fonte numéro + titre
  {\thesubsubsection}% numéro
  {0.5em}% espace entre le numéro et le titre
  {{\large\color{chapterLACL}▶}\hspace{0.5em}}% fonte titre
