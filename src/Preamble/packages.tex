% Vim: set spelllang=fr:
% PAQUETS
%\usepackage{amsfonts}
\usepackage{amsmath}
\usepackage{amssymb}
\usepackage{amsthm}
\usepackage{array}
\usepackage[backend=biber,language=french]{biblatex}
\usepackage{booktabs}
\usepackage{color}
%\usepackage{comment}
\usepackage{csquotes}
%\usepackage{fancybox}
%\usepackage{fancyhdr}
\usepackage{float}
%\usepackage{flushend}
%\usepackage[left=2cm, top=2cm, right=2cm, bottom=2cm]{geometry}
\usepackage{graphicx}
\usepackage[unicode]{hyperref}
\usepackage{iftex}
%\usepackage{listings}
\usepackage{minitoc}
%\usepackage{multicol}
\usepackage{multirow}
\usepackage{setspace}
\usepackage{tikz}
\usepackage{xcolor}
\usepackage{xspace}
\usepackage{xpatch}
\usepackage{wrapfig}

% Chap. théorie des jeux
\usetikzlibrary{automata,calc,shapes}
\theoremstyle{definition}
\newtheorem{theorem}{Théorème}                                                    %
\newtheorem{lemma}[theorem]{Lemme}
\newtheorem{definition}{Définition}
\newtheorem{remark}{Remarque}

% FONTES & LANGAGES
\ifXeTeX\else
\ifLuaTeX\else
\begingroup
  \errorcontextlines=-1\relax
  \newlinechar=10\relax
  \errmessage{^^J
  *************************************************^^J
  * (Lua|Xe)TeX is required to compile this document.^^J
  * Sorry!^^J
  *************************************************}%
\endgroup
\fi\fi

\ifLuaTeX
    \usepackage{luatextra}
    \usepackage{microtype}
\else
    \usepackage{fontspec}
\fi
\usepackage{polyglossia}

\setdefaultlanguage{french}
\setotherlanguage{english}
\defaultfontfeatures{Ligatures=TeX}
%\defaultfontfeatures{Ligatures={NoRequired,NoCommon,NoContextual}}
\setmainfont[%
        UprightFeatures={StylisticSet=2},%
        BoldFeatures={StylisticSet=1},%
        ItalicFeatures={StylisticSet=1},%
        %BoldItalicFeatures={StylisticSet=1},%
    ]{Linux Libertine O}
\setsansfont[%
        UprightFeatures={StylisticSet=2},%
        BoldFeatures={StylisticSet=1},%
        ItalicFeatures={StylisticSet=1},%
        BoldItalicFeatures={StylisticSet=1},%
    ]{Linux Biolinum O}
\setmonofont{Inconsolata}

\makeatletter
    \definecolor{blueLACL}{RGB}{\@blueLACL}
\makeatother

% HYPERSETUP
\hypersetup{%
    linktoc=all,
    %backref=true,
    breaklinks,
    colorlinks,
    linkcolor     = blueLACL,
    citecolor     = blueLACL,
    urlcolor      = blue,
    baseurl       = http://,
    pdfpagelayout = OneColumn, % pdfpagelayout = SinglePage
    pdfstartpage  = 1,
    pdfcreator    = {Vim, \LaTeX{}, and a lot of packages!},
    %pdfproducer   = {\LaTeX{}},% Contient automatiquement le compilateur
    bookmarksopen = true,
    bookmarksdepth= 2,
    pdfauthor     = {Quentin MONNET},
    pdftitle      = {Manuscrit de thèse},
    pdfsubject    = {Réseaux de capteurs et dénis de service},
    pdfkeywords   = {Thèse~; Manuscrit}
}

% MINITOC
\mtcsetdepth{minitoc}{1}
\mtcsettitle{minitoc}{}
%\mtcselectlanguage{french}

% BIBLIO
\addbibresource{Back/biblio.bib}
\addbibresource{Back/thjeux.bib}
\addbibresource{Back/studiauniv.bib}
% \chapter instead of \chapter*
\defbibheading{bibliography}[\bibname]{\chapter{#1}}%\markboth{#1}{#1}}

% OPTIONS
\renewcommand{\labelitemi}{\textbullet}
