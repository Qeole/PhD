% vim: set spelllang=fr foldmethod=marker:
\section{Organisation de la thèse}

\newcommand\chappar[1]{%
    \paragraph{\hyperref[chap:#1]{%
        Chapitre~\ref*{chap:#1}: \nameref*{chap:#1}%
}}}

\chappar{st}
Les réseaux de capteurs reposent sur des appareils faibles en capacités de calcul, en mémoire, ainsi qu'en termes d'énergie disponible.
Leur déploiement, leurs communications, leur organisation autonome, ainsi que les tâches qui leur sont assignées doivent composer avec ces limites.
Ce chapitre s'attache donc à présenter l'architecture basique des capteurs et des réseaux qui en sont constitués, à donner quelques exemples d'applications, puis à introduire les problématiques principales liées à ces réseaux: contraintes en ressources, mobilité éventuelle, sureté et résilience, et clusterisation hiérarchique du réseau en forment les grands axes.

\chappar{ea}
La sécurité informatique est transversale à l'ensemble des composantes de ce domaines.
Pour les réseaux de capteurs, les mécanismes de sécurité sont soumis aux mêmes restrictions en ressources que les autres protocoles.
Pour autant, la sécurité ne doit pas être négligée; elle est même indispensable pour toutes les applications critiques.
Dans ce chapitre seront développées les principales problématiques qu'elle regroupe: confidentialité et authentification sont parmi les plus importantes, et ont fait l'objet, dans les réseaux de capteurs, de propositions spécifiques.
La résistance aux attaques par déni de service est également un aspect important, qui bien sûr est central dans cette thèse.
Il est donc abordé en plusieurs temps: après la catégorisation puis la description des attaques majeures connues, les méthodes de prévention, de détection puis de réaction sont développées.

\chappar{sa}


\chappar{se}

\chappar{sd}

\chappar{cp}

\chappar{tj}
