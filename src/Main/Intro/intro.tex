% vim: set spelllang=fr foldmethod=marker:
\section{Réseaux de capteurs sans fil et déni de service}

\lettrineh{L}{a lutte} contre les attaques de type «déni de service» dans les réseaux de capteurs sans fil est le fil directeur des travaux présentés dans cet ouvrage.\linebreak
Les réseaux de capteurs sont constitués de petits appareils ---~les capteurs~--- équipés d'un module de communication sans fil.
Dispersés dans l'environnement à étudier, ces capteurs sont chargés de réaliser des mesures physiques, de les convertir en un signal numérique, et de les rapatrier pour un traitement plus approfondi à une station de base, qui fait office d'interface entre le réseau et l'utilisateur.
Cette collecte d'information est soumises aux contraintes en ressources des capteurs, dont les capacités en calcul, en mémoire, et dont les réserves d'énergie disponible (sous forme de batteries) sont limitées.

Mais les protocoles déployés ont su être adaptés: les capteurs sont aujourd'hui utilisés dans une multitude d'applications dans des domaines aussi variés que l'environnement, la santé, l'urbanisme, les transports, la domotique, et bien sûr le domaine militaire.
Liés à la fois aux avancées technologiques et à l'émergence de nouveaux usages dans l'exploitation des données, les concepts de «transports intelligents», de «villes intelligentes» ou d'«Internet des objets» se développent peu à peu et semblent promettre un usage de plus en plus intensif des réseaux de capteurs sans fil.

Toutefois, ces réseaux particuliers introduisent leur lot de problématiques: à côté des contraintes fortes en ressources se pose la question de la sécurité, dont la mise en place est une nécessité absolue pour les applications médicales ou militaires par exemple.
Alors que les mécanismes d'authentification et de chiffrement font appel, dans la plupart des systèmes informatiques, à des protocoles cryptographiques couteux en ressources, il a fallu adapter les mécanismes au monde des capteurs.
Mais la sécurité comporte d'autres volets, et la disponibilité des réseaux, suivant le contexte, peut s'avérer tout aussi essentielle.

En résumé, le problème se pose ainsi: comment prévenir, ou à défaut comment détecter puis contourner, tout en économisant les ressources des capteurs, une action d'origine malveillante qui viserait à mettre le réseau hors service?

Cette thèse s'appuie sur l'attribution d'un rôle de surveillance à certains des capteurs, chargés de détecter des comportements hostiles au sein du réseau, et de prévenir leurs pairs lorsqu'une attaque est détectée.

C'est notamment le processus de sélection dynamique de ces capteurs de surveillance qui fait l'objet d'une étude poussée.
Plusieurs solutions sont proposées: une sélection aléatoire, permettant d'obtenir statistiquement une bonne couverture du réseau; une sélection basée sur l'énergie résiduelle des capteurs, dont l'avantage est d'offrir une meilleure répartition de la charge (en matière de consommation énergétique) dans le réseau; et enfin un processus d'élection démocratique, basé sur des scores de réputation, qui améliore encore la sécurité du dispositif.

Mais en sus des algorithmes concrets, il est parfois utile de pouvoir représenter un processus sous un aspect plus formel.
Validées par des simulations, certaines des méthodes proposées sont également modélisées à l'aide de chaines de \textsc{Markov} ou de réseaux de \textsc{Petri} particuliers.
En fin d'étude, les jeux quantitatifs sont utilisés pour caractériser à plus haut niveau le système composé d'un capteur corrompu et des capteurs de surveillance.
\pagebreak %%%%%%%%%%%%%%%%%%%%%%%%%%%%%%%%%%%%%%%%%%%%%%%%%%%%%%%%%%%%%%%%%%%%

Le but des travaux présentés dans cette thèse est donc de proposer un ensemble de méthodes de détection et de réaction efficaces aux attaques par déni de service, tout en économisant ou en répartissant au mieux la consommation énergétique des capteurs pour prolonger le plus possible la durée de fonctionnement du réseau.
Les outils de modélisation fournis permettent à la fois de valider ces méthodes, de mieux les comprendre, et de servir de base, dans le futur, pour la conception de mécanismes toujours plus performants.

\section{Organisation de la thèse}

\newcommand\chappar[1]{%
    \paragraph{\hyperref[chap:#1]{%
        Chapitre~\ref*{chap:#1}: \nameref*{chap:#1}%
}}}

\chappar{st}
Les réseaux de capteurs reposent sur des appareils faibles en capacités de calcul, en mémoire, ainsi qu'en termes d'énergie disponible.
Leur déploiement, leurs communications, leur organisation autonome, ainsi que les tâches qui leur sont assignées doivent composer avec ces limites.
Ce chapitre s'attache donc à présenter l'architecture basique des capteurs et des réseaux qui en sont constitués, à donner quelques exemples d'applications, puis à introduire les problématiques principales liées à ces réseaux: contraintes en ressources, mobilité éventuelle, sureté et résilience, et clusterisation hiérarchique du réseau en forment les grands axes.

\chappar{ea}
La sécurité informatique est transversale à l'ensemble des composantes de ce domaine.
Pour les réseaux de capteurs, les mécanismes de sécurité sont soumis aux mêmes restrictions en ressources que les autres protocoles.
Pour autant, la sécurité ne doit pas être négligée; elle est même indispensable pour toutes les applications critiques.
Dans ce chapitre sont développées les principales problématiques qu'elle regroupe: confidentialité et authentification sont parmi les plus importantes, et ont fait l'objet, dans les réseaux de capteurs, de propositions spécifiques.
La résistance aux attaques par déni de service est également un aspect important, qui bien sûr est central dans cette thèse.
Il est donc abordé en plusieurs temps: après la catégorisation puis la description des attaques majeures connues, les méthodes de prévention, de détection puis de réaction sont développées.

\chappar{sa}
Une méthode de détection des attaques par déni de service consiste à assigner à certains capteurs (appelés \cnst) un rôle de surveillance au sein du réseau.
Combinée à une partition en clusters du réseau, elle permet d'observer efficacement le trafic pour repérer les comportements anormaux.
Se pose alors la question du renouvellement du rôle de surveillance: changer de \cnst au cours du temps permet de répartir la consommation engendrée par cette tâche de surveillance, et par là de prolonger la durée de vie du réseau dans son ensemble.

Le processus de sélection peut être effectué simplement en choisissant un nombre aléatoire et en le comparant à un seuil de probabilité: c'est la méthode, simple à mettre en œuvre, utilisée dans ce chapitre.
Les simulations permettent d'observer l'amélioration importante obtenue sur la répartition de la charge énergétique et valident l'emploi d'un renouvellement dynamique des \cnst.

Afin de pouvoir l'analyser de façon plus formelle, ce renouvellement est également modélisé sous forme de chaines de \textsc{Markov} à temps continu, mais cet outil a ses limites; pour les contourner, une seconde méthode formelle est employée: il s'agit des réseaux de \textsc{Petri} stochastiques généralisés étendus, qui permettent une représentation efficace du modèle.
De plus, il est possible d'en dériver des formules de logique stochastique avec automates hybrides, et des exemples de ces formules sont fournis. Ils permettent à leur tour de réutiliser le modèle obtenu par le biais de techniques de \textit{model checking} pour vérifier diverses propriétés sur le système.

\chappar{se}
Une deuxième façon de renouveler les \cnst consiste à considérer leur énergie résiduelle à l'instant où le processus de sélection est exécuté, de sorte que les capteurs possédant les plus grandes réserves d'énergies se voient attribuer la tâche la plus couteuse.
Mais il ne faut pas oublier que dans un contexte de détection des attaques, le rôle de \cnt est particulièrement intéressant à endosser pour un capteur compromis, puisqu'il peut l'aider à éviter de se faire repérer.
S'il n'est pas possible d'empêcher un capteur de «mentir» sur la valeur de son énergie résiduelle, des méthodes sont appliquées en revanche pour détecter ces tentatives frauduleuses d'accès au rôle.
Elles passent par la nomination de capteurs (les \vnst) chargés de vérifier l'évolution de la consommation des \cnst afin de détecter les tentatives de fraude.
Un second mécanisme est également mis en place pour assurer la bonne couverture du réseau par les \cnst, menacée par l'aspect déterministe du processus de sélection basé sur l'énergie restante des capteurs.
Des simulations sont une nouvelle fois utilisées pour mettre en avant le gain supplémentaire apporté à la répartition de la charge en énergie.

\chappar{sd}
Plutôt que de se concentrer uniquement sur la répartition de la charge énergétique, il est possible d'envisager une approche cherchant à maximiser la sécurité du réseau.
La troisième méthode proposée pour la sélection des \cnst consiste à les faire élire par les capteurs voisins, en fonction d'un score de réputation établi au fil du temps, \cad en fonction de la confiance accumulée par chaque capteur pour ses «bonnes» actions (ou du moins l'absence de «mauvaises» actions).
Les nœuds «élus» sont placés dans une liste de candidats de confiance.
Le score final peut être ajusté en fonction d'autres critères, en reprenant par exemple l'énergie résiduelle, de sorte que les candidats «idéaux» au regard de la somme des critères retenus se voient attribuer le rôle de \cnt.
Cette troisième méthode est à son tour évaluée, puis comparée à celles des deux chapitres précédents, à travers de nouvelles simulations.
Cette comparaison permet d'établir, en fonction des performances recherchées (ou du critère à favoriser), quelles sont les meilleures solutions à déployer pour la sélection des \cnst dans un réseau de capteurs sans fil.

\chappar{tj}
Le dernier chapitre technique revient sur des aspects de modélisation, cette fois-ci en se basant sur la théorie des jeux.
La représentation ne porte plus sur le processus de sélection pour la surveillance, mais sur les interactions entre \cnst et capteurs compromis.
Le système est modélisé sous forme de jeux quantitatifs, qui tiennent compte à la fois de l'énergie des capteurs et d'une valeur de gain incrémentée à chaque transmission réussie; les graphes correspondants, les conditions de victoire sont fournis, et un exemple est développé sur le premier tour de jeu.
La résolution générale des jeux produits est indécidable, mais elle est fournie pour un sous-ensemble d'objectifs constitués de conjonctions de littéraux.
Cette modélisation peut être vue comme une ouverture sur des travaux futurs, qui permettraient d'établir une méthode de détection plus efficace des capteurs compromis en évaluant leurs performances selon un ensemble de stratégies logiques.

\chappar{cc}
Une synthèse des travaux abordés est réalisée avant de clôturer l'ouvrage: les différents mécanismes de sélection des \cnst, leurs principales caractéristiques, leurs cas d'utilisation sont passés en revue.
Il est fait de même pour les différents outils de modélisation proposés.
Les réseaux de \textsc{Petri} utilisés, la logique stochastique, les jeux quantitatifs ont permis de créer des représentations formelles des systèmes étudiés, qui ne demandent qu'à être adaptées à d'autres cas d'utilisation.
Mais d'autres pistes concernant de futurs angles d'approche sont présentées à la fin de cette conclusion, et offrent des perspectives sur les potentielles évolutions des travaux présentés dans cet ouvrage.
