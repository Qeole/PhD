% vim: set spelllang=fr foldmethod=marker:
\section{Processus d'élection démocratique des \cns}\label{sd:sec:proposal}

Ce troisième processus de sélection se base lui aussi sur un renouvellement dynamique.
Chaque renouvellement sera appelé une \emph{itération} du processus.

Pour efficacement celui-ci, nous allons adopter les notations suivantes:
\begin{itemize}
    \item $\mathcal{N}$ désigne l'ensemble des capteurs du cluster (à l'exception du \ch);
    \item $\mathcal{CN}$ désigne l'ensemble des \cns du cluster à l'instant considéré;
    \item $i$ est un index utilisé pour parcourir $\mathcal{N}$;
    \item soit $RE_k$ un tableau contenant l'énergie résiduelle des nœuds annoncée par les \cns au \ch à la $k$\textsuperscript{e} itération.
    \item soit $Obs_k$ un tableau contenant les observations réalisées par le \cn~$j$ sur les communications de ses voisins.
\end{itemize}

Le processus d'élection débute, avant sa première itération, par une phase d'initialisation.
Viennent ensuite les itérations elle-même qui marquent le renouvellement de l'attribution du rôle de \cn, tant que le réseau est en fonctionnement.

\subsection{Phase d'initialisation}

Les étapes suivantes sont réalisées une fois le cluster formé:
\begin{itemize}
    \item chaque nœud~$i$ du cluster envoie au \ch la valeur de son énergie résiduelle; le \CH l'enregistre dans un tableau spécifique, dénoté~$RE_0[i]$;
    \item chaque nœud agit temporairement comme un \cn, et commence à surveiller le trafic de ses voisins. Il conserve en parallèle le fonctionnement d'un nœud normal (en termes de mesure physiques et de collecte de données), sans quoi il n'y aurait aucun trafic à observer. Cette étape a pour but d'essayer de repérer, dès le départ, d'éventuels nœuds compromis, pour les écarter de la liste des candidats au rôle de \cn;
    \item une fois la durée de cette première étape d'observation écoulée, le processus d'élection démocratique peut commencer. La valeur du compteur d'itérations~$k$ est passée à $1$.
\end{itemize}

\subsection{Sélection des \cns}

Chaque itération~$k$ se décompose en étapes, qui sont détaillées ici:
\begin{enumerate}
    \item la phase « active » de collecte de données a lieu. Les capteurs non sélectionnés amassent puis communiquent leurs données, sous la surveillance des \cns. Ces derniers journalisent le nombre, la taille et la date des paquets envoyés par chacun de leur voisins, et signalent les éventuels comportements suspects au \ch;
    \item à la fin de l'itération~$k$, chaque nœud~$i$ envoie son énergie résiduelle (dénotée~$RE_k[i]$); chaque \cn~$j$ envoie en plus un tableau~$Obs_k[j]$ contenant les observations effectuées sur le débit d'émission de ses voisins au cours de l'itération;
    \item pour chaque nœud~$i$, le \ch réalise l'analyse suivante:
        \begin{itemize}
            \item en se basant sur un modèle mathématique adapté, le \CH calcule la consommation énergétique « effective » $ECe$ relative au débit moyen\footnote{ou maximum? Mais attaques possibles. Ou maximum pondéré?} $\{Obs_k[j][i]\}|_{j\in \mathcal{CN}}$ observé par les \cns~$j$ voisins de~$i$ au cours de l'itération,
            \item le \CH calcule alors la consommation « annoncée » pour la période~$k$, en prenant la différence des dernières valeurs annoncées par~$i$: $ECd=RE_{k-1}[i] - RE_k[i]$,
            \item si $|ECa-ECe|\leq\epsilon$ (où $\epsilon$ est la marge d'erreur tolérée), alors le nœud~$i$ est déclaré « sain », et il est ajouté à l'ensemble~$SEN$ des nœuds éligibles pour la prochaine sélection des \cns,
            \item sinon, le nœud est placé dans la liste~$SSN$ des nœuds suspects,
            \item soit $SSN[i]$ le nombre de fois où le nœud~$i$ a été déclaré suspect depuis le déploiement du réseau. Si $SSN[i] \ge \mathrm{seuil}$, alors le capteur est déclaré compromis et devient virtuellement exclu du cluster; si à l'inverse le capteur avait systématiquement adopté un comportement honnête, il peut s'agir d'un faux positif, et le nœud n'est pas définitivement exclu. Il conserve la possibilité de rejoindre l'ensemble~$SEN$ au terme de prochaines itérations;
        \end{itemize}
    \item une fois l'étape~3 terminée, le \CH sélectionne les \cns parmi l'ensemble~$SEN$ des nœuds éligibles. Le choix est réalisé de telle manière que chaque capteur, \cns compris, soit contrôlé par au moins deux \cns distincts. Ainsi, un nœud compromis qui aurait néanmoins réussi à se faire élire en tant que \cn (avant de démarrer son attaque, par exemple), mais se comportant de façon malicieuse sur l'envoi de données, sera détecté;
    \item enfin l'itération suivante démarre: le compteur~$k$ est incrémenté, et l'on retourne à l'étape~1.
\end{enumerate}

\subsection{Modèle mathématique pour la consommation en énergie}

Tout comme dans le chapitre précédent, la solution présentée ici implique des estimations sur la consommation en énergie des capteurs du cluster.
Nous proposons à nouveau pour ces calculs d'utiliser le modèle de diffusion de Rakhmatov et Vrudhula, détaillé au \chapref{se}, \ssref{se:rakvru-formula}.

\subsection{Sélection des \cns parmi l'ensemble $SEN$}

À l'étape~3 du processus d'élection, les \chs choisissent les \cns pour l'itération à venir parmi tous les nœuds placés dans la liste~$SEN$.
Les critères retenus pour effectuer ce choix peuvent être très variés.
Par exemple, il est possible de tirer aléatoirement des \cns parmi les nœuds de~$SEN$, jusqu'à ce que:
\begin{itemize}
    \item l'on ait assez de \cns (selon le seuil déterminé par l'utilisateur du réseau);
    \item et que tous les nœuds du cluster soient couverts par au moins deux \cns, comme expliqué plus haut.
\end{itemize}
D'autres façons d'effectuer ce choix reviendraient par exemple à prendre en compte un ou plusieurs critères parmi la liste suivante:
\begin{itemize}
    \item l'énergie résiduelle des nœuds;
    \item l'index de connectivité (le nombre de voisins directs des nœuds);
    \item la puissance du signal;
    \item \etc
\end{itemize}
Plusieurs de ces critères peuvent être pondérés et combinés de façon à obtenir une note pour chaque nœud, comme par exemple dans l'\equaref{se:eqn:score}.
\begin{equation}
    \label{se:eqn:score}
    s_k[i] = (\alpha \times RE_k[i]) + (\gamma \times ci_k[i]) + (\delta \times sp_k[i]) + (\zeta \times nsl_k[i])
\end{equation}
où:
\begin{itemize}
    \item $s_k[i]$ représente la note du nœud~$i$ pour l'itération~$k$;
    \item $RE_k[i]$ reste l'énergie résiduelle pour le même nœud et la même itération;
    \item $ci_k[i]$ est l'index de connectivité de~$i$;
    \item $sp_k[i]$ est la puissance moyenne du signal enregistrée par les voisins du nœud~$i$;
    \item $nsl_k[i]$ est la valeur maximale entre un seuil prédéterminé et le nombre d'itérations antérieures à~$k$ depuis que le nœud a été désigné \cn pour la dernière fois;
    \item $\alpha$, $\gamma$, $\delta$ et $\zeta$ sont des constantes définies par l'utilisateur.
\end{itemize}
Cette formule pourrait être utilisée pour classer les nœuds de la liste~$SEN$ de telle manière que le \ch puisse choisir les meilleurs \cns possibles par rapport aux critères pris en compte.

La détermination des critères à retenir et des poids à appliquer revient à l'utilisateur du réseau; en pratique, il faudrait les adapter par rapport à l'application qui sera faite du réseau de capteurs, tout en tenant compte des conditions particulières de l'environnement dans lequel celui-ci sera déployé.
Dans un réseau constitué de nœuds statiques, par exemple, l'index de connectivité des capteurs, de même que la puissance de leur signal, ne sont pas censés subir de modifications sensibles entre deux itérations consécutives.
En conséquence, les poids de ces deux critères devraient être modérés (par rapport aux autres poids utilisés dans la formule) afin d'éviter de désigner les mêmes capteurs à chaque itération.
Dans un réseau clusterisé, où tous les nœuds peuvent atteindre leur \CH en un seul saut, l'index de connectivité ou la puissance du signal ne constituent peut-être pas des critères pertinents pour la formule de sélection (bien que, là encore, cela puisse dépendre de l'application déployée).
Mais même si nous travaillons principalement dans des réseaux clusterisés dans cet ouvrage, il est bon de se rappeler que tous les réseaux de capteurs ne font pas nécessairement appel à de tels mécanismes.
Et que nombre d'entre eux comportent également des nœuds mobiles.
Dans de tels environnements, l'évaluation de la densité des nœuds ou de la qualité des liens dans les zones où sont situés les \cns candidats devient plus intéressante, car il s'agit souvent d'indicateurs précieux sur les performances du réseau.
Toujours sur le même propos, la contrainte fixée consistant à faire en sorte que chaque nœud soit surveillé par au moins deux \cns peut éventuellement être aménagée.
S'il s'agit en principe d'une bonne assurance pour détecter les nœuds compromis dans des clusters où tous les nœuds sont réunis à portée d'émission de leur \ch, cette contrainte pourrait s'avérer beaucoup plus lourde pour d'autres topologies.
Ainsi dans un réseau en étoile, par exemple, où tous les capteurs se retrouveraient sur des branches et n'auraient que deux voisins (l'un plus près, l'autre plus loin de la station de base), la mise en application de la contrainte pourrait avoir pour effet de désigner tous les nœuds du réseau comme \cns.
