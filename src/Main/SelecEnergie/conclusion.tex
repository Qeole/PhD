\section{Conclusion}

\cns are used in clustered \wsns to monitor traffic of the nodes and to detect \dos attacks (\eg flooding, black hole attacks).
In this paper, we have proposed a new method to dynamically elect those \cns, based on their residual energy.
The aim of the proposed selection algorithm is to provide a better load balancing in the cluster.

We have addressed several issues related with the use of a deterministic selection.
Compromised nodes trying to systematically take over the \cn role are forced to abandon it for later cycle, or get detected, by \vns.
The \vn role is a new role we introduced to survey the \cns by matching their announced energy consumption with a theoretical model.
The issue of areas of the cluster uncovered by \cns, depending of the activity in the cluster, is addressed by enforcing covering of the whole cluster: the \ch is to designate additional \cns if needed.
Working with clusters ensures a good scalability of the solution.
It is also flexible, as \cns can endorse various trust-based model, and monitoring rules can be set to fight against several types of \dos attacks.
And the use of \vns is resilient to a small percentage of compromised \vns (depending on parameters set by user).

The results we have obtained through simulations show that even though using our simulation causes a higher global consumption of energy in the cluster, it provides a better load repartition between sensors.

Future works include improvements of our solution by adding monitoring of the \ch, as well as modeling a cluster with areas of different activity levels.
Also we would especially like to study the impact of the percentage of designated \vns on global energy consumption.
