% vim: set spelllang=fr foldmethod=marker foldlevel=1:
\subsection{Détecter les attaques: systèmes de détection d'intrusion}
%{{{1

    \subsubsection{Différents systèmes de détection d'intrusion}
%{{{2
La détection des attaques de \dds passe par la mise en place d'un système spécifique, apte à collecter des indices permettant de déterminer si une attaque a lieu et, si possible, quelle en est l'origine.
On parle de « \idss », ou \IDS pour \textit{Intrusion Detection System}.
Pour être précis, il convient de relever que ces systèmes sont utilisés pour la détection de tous types d'attaques, y compris lorsqu'il ne s'agit pas de \dds.
Les usages et les spécifications des \IDS sont donc multiples: la \figref{ea:fig:idsmap} résume les différentes catégories existantes, que nous allons brièvement décrire.
Il existe différentes façon de les distinguer.
\nomenclature{IDS}{\textit{Intrusion Detection System}}
% vim:se ft=tex:
\begin{figure}[!ht]
    \caption{Critères de classification pour les systèmes de détection d'intrusion}\label{ea:fig:idsmap}
    \medskip
    \centering
    \begin{tikzpicture}[%
            small mindmap, every node/.style=concept, concept color=chapterLACL!50,
            grow cyclic,
            level 1/.append style={level distance=3.8cm,sibling angle=51.5},
            level 2/.append style={level distance=1.7cm,sibling angle=45}%
        ]

        \node {Syst\`emes de d\'etection d'intrusion}
        child { node {Provenance de l'attaquant}
            child { node {Externe} }
            child { node {Interne} }
        }
        child { node {Type d'intrusion}
            child { node {Tentatives d'acc\`es au r\'eseau} }
            child { node {Usurpation d'identit\'e} }
            child { node {Acc\`es non authoris\'e} }
            child { node {Fuite de donn\'ees} }
            child { node {Blocage des ressources} }
            child { node {Destruction des ressources} }
        }
        child { node {Infrastructure}
            child { node {Plate} }
            child { node {Clusteris\'ee} }
        }
        child { node {M\'ethode de d\'etection}
            child { node {Anomalies} }
            child { node {Signatures} }
            child { node {Sp\'ecifications} }
        }
        child { node {Fr\'equence d'usage}
            child { node {En continu} }
            child { node {P\'eriodique} }
        }
        child { node {Sujet audit\'e}
            child { node {Machine} }
            child { node {R\'eseau} }
            child { node {Syst\`eme hybride} }
        }
        child { node {Gestion des calculs}
            child { node {Centralis\'es} }
            child { node {Chacun pour soi} }
            child { node {Distribu\'es} }
            child { node {Hi\'erarchiques} }
            child { node {Agents mobiles} }
        };
    \end{tikzpicture}
\end{figure}


        \paragraph{Selon l'origine de l'attaque}
Un \ids peut se concentrer sur les attaques provenant de l'\textbf{extérieur} du réseau, menées par un attaquant qui chercherait à brouiller les fréquences utilisées ou bien, justement, à pénétrer dans le réseau.

À l'inverse, d'autres systèmes vont surveiller l'\textbf{activité interne} du réseau, afin de détecter d'éventuels nœuds corrompus, soit qu'ils cherchent à accaparer\index{comportement cupide!accaparement} les ressources pour leur propre compte, soit qu'ils cherchent à détruire les ressources du réseau.

        \paragraph{Selon l'objet de la surveillance}\label{ea:sss:hids}
Un \IDS peut avoir pour tâche de surveiller l'\textbf{activité du réseau}, en analysant le trafic: le nombre de paquets transmis, leur origine ou leur destination, leur contenu, le taux de succès des transmissions sont autant d'indices à collecter pour déterminer si le réseau fonctionne dans des conditions normales.

En parallèle, d'autres systèmes peuvent être orientés vers la \textbf{sécurité de l'hôte}\index{securité@sécurité!sécurité système} sur lequel ils sont exécuté.
Leur but est de détecter d'éventuelles tentatives de compromission de la machine.
Les opérations menées à cette fin peuvent comprendre:
\begin{itemize}
    \item la vérification de l'\integrite du système (par exemple en calculant périodiquement un \idx{condensat} des fichiers exécutables critiques, afin de les comparer avec le \idx{condensat} de leur version d'origine);
    \item la surveillance des journaux concernant les tentatives de connexion, afin de détecter des échecs répétitifs d'accès au système;
    \item la détection d'une activité inhabituelle au niveau de l'activité du processeur, des allocations mémoires, ou des entrées sorties, qui risqueraient de pénaliser le système d'exploitation en termes de performance et de consommation énergétiques. Cette surveillance peut être menée en temps réel, ou via la journalisation des évènements système.
\end{itemize}
On parle en anglais de NIDS\index{système de détection d'intrusion!NIDS} (\textit{Network Intrusion Detection System}) ou de HIDS\index{système de détection d'intrusion!HIDS} (\textit{Host based Intrusion Detection System}) pour les \IDS orientés réseau ou machine, respectivement.
De nombreux \IDS cumulent à la fois des mécanismes de surveillance du réseau et du comportement de leurs voisins, et une surveillance des évènements système survenant sur leur hôte~\cite{BMS13}.

        \paragraph{Selon le type d'intrusion à détecter}
Les \IDS sont utilisés pour lutter contre tous types d'« intrusions » (ou d'attaques dans le sens plus général), pas seulement contre le \dds, bien que ce soit le point qui nous préoccupe le plus dans cette thèse~\cite{BMS13}.
Ainsi ils peuvent être utilisés afin de détecter:
\begin{itemize}
    \item des \textbf{tentatives d'accès au réseau} de la part d'un attaquant externe (qui chercherait à passer par exemple par la compromission d'un nœud ou par l'obtention de matériel cryptographique\index{cryptographie}); cette étape n'est souvent qu'un prélude à d'autres attaques (de \dds notamment);
    \item un \textbf{accès à des ressources non autorisées}, lorsque qu'un capteur tente par exemple de procéder à certaines opérations dans un cluster qui n'est pas le sien;
    \item les \textbf{fuites de données}, qui peuvent être difficiles à détecter si l'attaque est passive (simple écoute des données circulant sur le réseau), mais peuvent être identifiées lors de l'extraction de ces données vers l'extérieur du réseau;
    \item les \textbf{comportements cupides} provoquant le blocage des ressources, par accaparement\index{comportement cupide!accaparement};
    \item les autres attaques de service, \cad les \textbf{comportements destructeurs}, visant à nuire au fonctionnement du réseau par l'anéantissement des ressources, qu'elles soient virtuelles (mise hors service des transmissions) ou bien physiques (épuisement de la batterie des capteurs).
\end{itemize}

        \paragraph{Selon les méthodes employées pour la détection}
Les méthodes utilisées pour déterminer l'existence et l'identité d'un attaquant sont nombreuses, et se divisent en trois catégories: la détection \textbf{basée sur les anomalies}\index{anomalie}, la détection \textbf{basée sur des signatures d'attaques}\index{signature d'attaque}, et la détection reposant \textbf{sur la vérification du respect des spécifications}~\cite{BMS13}.
Ces catégories, et les principales méthodes qu'elles regroupent, sont détaillées plus bas en \sssref{ea:sss:detect}.

        \paragraph{Selon l'infrastructure du réseau}
Si de nombreux \IDS ne se préoccupent pas de savoir quelle est la configuration exacte du réseau, et sont élaborés sur des architectures « plates », \textbf{sur un seul niveau}, d'autres en revanche sont spécifiquement construits avec l'idée de tirer profit d'une \textbf{architecture clusterisée}\index{clusterisation!architecture clusterisée}.
La partition du réseau en clusters va de fait permettre la mise en place d'algorithmes différents, le plus souvent distribués au sein des clusters~\cite{SJ11}.

Il est également intéressant de noter que la clusterisation\index{clusterisation} en soi d'un réseau est un mécanisme efficace pour limiter la portée de la plupart des attaques de \dds, dont les dommages seront en principe circonscrit au cluster dans lequel elles sont menées.
Une exception concerne toutefois les attaques qui sont menées avec succès contre les \ch (notamment lorsque celui-ci se voit compromis), puisque ce peut être alors le cluster entier qui se retrouve hors service.

        \paragraph{Selon la \idx{distribution} des calculs}
La détection passe par la collecte puis le traitement d'indices sur le fonctionnement des nœuds ou du réseau global~\cite{BMS13}.
Collecte comme traitement peuvent être réalisés par différenties entités ou associations d'entités, à savoir:
\begin{itemize}
    \item le traitement, les calculs peuvent être \textbf{centralisés} et réalisés seulement par la \sdb, à laquelle les capteurs font remonter les informations brutes relevées sur le fonctionnement du réseau. Dans certains cas extrêmes, même la collecte d'indices n'est réalisée que par la \sdb;
    \item chaque capteur peut implémenter son propre \ids: « \textbf{chacun pour soi} », quitte à garder pour lui ses conclusions sur le statut des autres nœuds. À l'inverse de la solution précédente, chaque capteur réalise alors de bout en bout les opérations de surveillance;
    \item la détection peut être menée par des \textbf{agents mobiles}\index{mobilité}, qui se déplacent de nœuds en nœuds dans un \rc autrement statiques (ou peu mobiles en comparaison avec les agents de surveillance). Ces agents mobiles\index{mobilité} disposent en principe de ressources plus importantes que les capteurs ordinaires. Cette configuration ne peut bien sûr pas être déployée pour tous les usages des \rcs;
    \item la collecte d'informations peut être \textbf{distribuée entre les nœuds}\index{distribution} situés sur un même niveau d'horizontalité dans l'architecture du réseau: cette configuration leur permet d'échanger des informations entre eux, notamment des indices de \idx{confiance}, et de déclencher des observations coordonnées lorsqu'un nœud « suspecte » un pair d'être compromis, sans en avoir la « certitude ». Lorsqu'un capteur est jugé définitivement compromis, l'information est partagée entre les nœuds. Dans certains systèmes, les calculs eux-mêmes peuvent être distribués;
    \item enfin, la distribution peut également être réalisée de façon \textbf{hiérarchique}, sur un plan vertical, dans les réseaux clusterisés\index{clusterisation!réseau clusterisé}. Il y a alors interaction entre les membres d'un cluster et leur \ch, qui se charge au minimum de la propagation des alertes, et parfois également des calculs à mener.
\end{itemize}

La gestion de la collecte et des calculs vont bien évidemment influencer l'opérateur du réseau sur le choix du type de méthode de détection (voir plus haut) à mettre en place.

        \paragraph{Selon la fréquence d'usage du système}
La plupart des \IDS sont prévus pour fonctionner \textbf{de manière continue} sur leur hôte, mais certaines solutions adoptent des comportement différents: le système de détection n'est exécuté que \textbf{de façon périodique}, ou bien se voit activé à la réception d'alerte (émanant de la \sdb par exemple, qui dispose pour sa part des ressources nécessaires au maintien en continu d'un \IDS)~\cite{BMS13}.

        \paragraph{Systèmes hybrides}
En pratique, pour un critère donné, la plupart des \IDS proposés se retrouvent dans plusieurs catégories à la fois: on retrouve ainsi beaucoup de systèmes hybrides surveillant tout à la fois l'hôte sur lequel ils tournent, et le trafic intercepté par le nœud.
De même, le type d'attaques recherchées par les systèmes sont généralement multiples, et portent souvent sur plusieurs couches protocolaires (en référence au modèle \tcpip).
Et pour en détecter un maximum, il n'est pas rare qu'un \IDS combine plusieurs méthodes de détection.
On voit ainsi souvent apparaitre en anglais le terme \textit{cross-layer IDS}\index{système de détection d'intrusion!\textit{cross-layer} IDS}, littéralement « IDS sur plusieurs couches »~\cite{BMS13}.
%2}}}

    \subsubsection{Les méthodes de détection}\label{ea:sss:detect}
%{{{2
Une fois le réseau déployé, et le \ids mis en place, les nœuds chargés de la surveillance du réseau analysent le trafic environnant et en tirent des déductions sur l'état des réseau.
Ces déductions résultent de l'application d'algorithmes précis sur les données.
Ces algorithmes sont de plusieurs sortes, que l'on regroupe dans les méthodes de détection basées sur des anomalies\index{anomalie}, sur des signatures d'attaques\index{signature d'attaque}, ou sur le non-respect de spécifications prédéfinies.

        \paragraph{Détection basée sur les anomalies\index{anomalie}}
Détecter les attaques selon les anomalies\index{anomalie} qui surviennent dans le réseau revient à établir, dans un premier temps, un modèle du fonctionnement normal du réseau~\cite{BMS13}.
Dans un second temps, la surveillance en elle-même est déclenchée, et si les variables observées dérogent sensiblement au modèle généré, une alerte est levée.
Ces méthodes permettent donc d'intervenir lorsqu'un problème survient effectivement dans le réseau, sans que son origine n'ait nécessairement été identifiée.
Elles ont pour avantage de s'adapter à de nouveaux types d'attaques, pour lesquelles elles n'avaient pas été imaginées au départ.
En revanche, elles peuvent être amenées à manquer certaines attaques connues dont les conséquences sont suffisamment discrètes.
Un second inconvénient existe: elles requièrent un réajustement constant du modèle par rapport à l'évolution normale du réseau, et la surveillance doit donc être maintenue tout à la fois pour détecter les attaques et pour mettre à jour le modèle.

La détection basée sur les anomalies\index{anomalie} regroupe trois sous-catégories.
\begin{itemize}
    \item La première est celle des \textbf{modèles statistiques}, qui enregistrent dans un premier temps des séries de valeurs pour des variables liées au bon fonctionnement du réseau, et cherchent par la suite à repérer les écarts importants à ces valeurs de référence. Les variables observées, directement liées au trafic, peuvent être considérées comme des variables indépendantes suivant des lois de \textsc{Gauss}, comme des variables corrélées, ou encore comme des séries temporelles.
    \item Une deuxième regroupe les modèles basés sur les \textbf{connaissances} modélisées du fonctionnement normal du réseau, et utilise un modèle généré en fonction des données et à l'aide d'outils tels que des langages de description ou des automates finis.
    \item Enfin, la détection peut être basée sur des \textbf{méthodes d'apprentissage}, où des modèles sont créés et actualisés \via des outils formels comme les processus de \textsc{Markov}, les réseaux bayésiens, la logique floue, les algorithmes génétiques, les réseaux de neurones, ou encore les modèles d'intelligence en essaim~\cite{BMS13}.
\end{itemize}

        \paragraph{Règles et signatures des attaques\index{signature d'attaque}}
Plutôt que d'attendre que des anomalies\index{anomalie} de fonctionnement, de performances, soient détectés, d'autres \IDS prennent les devants et mettent en place un système de surveillance du voisinage en établissant un ensemble de règles à ne pas transgresser.
Ces règles sont construites à partir de la « signature » des attaques connues\index{signature d'attaque} \cad, pour une attaque donnée, à partir du comportement type adopté par un nœud qui tenterait de mener cette attaque~\cite{BMS13}.

Les règles peuvent être nombreuses et variées, en voici quelques exemples représentatifs.\index{signature d'attaque}
\begin{itemize}
    \item \textbf{Règle sur la fréquence des envois}: le délai entre deux paquets consécutifs, ou plus généralement la fréquence des messages émis sur un intervalle de temps par le nœud surveillé, ne doivent pas dépasser un certain seuil (minimal si l'on parle de délai, maximal pour une fréquence). Cette règle empêche un capteur de procéder à des attaques par \deluge, ou bien d'accaparer\index{comportement cupide!accaparement} le médium de transmission, sans être détecté.
    \item \textbf{Règle de retransmission}: les messages reçus par un nœud relais, mais à destination d'un autre nœud, doivent impérativement être retransmis par ce relais au nœud suivant sur la route définie pour ces paquets, ceci afin de détecter les attaques de type \textit{blackhole} ou \textit{sinkhole}.
    \item \textbf{Règle sur le délai de retransmission}: la retransmission en question par le relais ne doit pas être effectuée après un certain délai, afin d'éviter des ralentissements importants dans le réseau (cette règle est très proche de la précédente, qui est bien obligée de se baser sur un délai limite pour considérer que le paquet ne sera pas retransmis).
    \item \textbf{Règle sur l'\integrite}: Le contenu du message retransmis par le relais ne doit pas être altérée. Cette règle est bien entendu inapplicable lorsque le relais est censé effectuer un traitement (agrégation, traitement sémantique\dots) sur ce contenu avant de le retransmettre. Les attaques visées sont principalement celles visant l'\integrite des paquets.
    \item \textbf{Règle sur la répétition}: un message ne peut être transmis par un nœud donné au plus qu'un certain nombre de fois. Si ce seuil est dépassé, le nœud peut être suspecté de vouloir influencer les résultats collectés par la \sdb. Si la répétition est espacée dans le temps, il peut aussi s'agir d'une tentative d'attaque par \idx{rejeu} de paquets.
    \item \textbf{Règle sur la portée de transmission}: les paquets « entendus » par les nœuds de surveillance ne devraient provenir que de nœuds voisins dans le réseau, et non de nœuds que l'on sait éloignés. Réciproquement, les nœuds voisins ne sont pas censés émettre avec une puissance dépassant un certain seuil. Cette règle peut permettre de détecter des nœuds qui tenteraient de produire une fuite de données vers l'extérieur du réseau, ou bien de mener des attaques dans des zones plus éloignées du réseau (notamment des attaques sur le \idx{routage}, en prétendant être voisin de nœuds distants, ou bien des attaques par trou de ver\index{trou!trou de ver}).
    \item \textbf{Règle sur les collisions}: pour lutter contre les tentatives de \idx{brouillage}, un seuil peut être instauré pour le nombre maximum d'échecs de transmission dû à des collisions\index{collision}. Si ces collisions sont trop fréquentes et que le capteur qui en est à l'origine peut être identifié, il est considéré comme compromis.
    \item \textbf{Règles sur les accès}: si le nombre de tentatives d'un nœud pour s'enregistrer et obtenir un accès à certaines ressources dépasse un certain seuil, une alarme est déclenchée. Cette règle peut notamment s'appliquer à la détection des intrusions au niveau machine, lorsqu'une entité tente et échoue plusieurs fois de se connecter sur le système.
\end{itemize}

Lorsque les règles sont transgressées, une « alarme » est déclenchée.
Suivant le modèle de l'\IDS, cette alarme peut n'être qu'une notification interne au capteur qui héberge le système, ou bien il peut partager l'information avec son entourage ainsi que, généralement, la \sdb.
Les mesures à prendre en guise de réactions seront traitées plus bas.

Les nœuds qui observent le comportement de leurs voisins et veillent à l'application des règles choisies pour le réseau sont désignés de plusieurs façons: on parle de nœuds de contrôle, de nœuds de surveillance, éventuellement de sentinelles; en anglais, ce seront le plus souvent des \textit{monitoring nodes} (nœuds moniteurs).
L'expression \textit{watch dogs}, pour « chiens de garde », est aussi régulièrement employée~\cite{RKKK13}.
Dans nos travaux, les nœuds chargés de cette surveillance sont appelés \cns.

La détection basée sur la signature des attaques\index{signature d'attaque} est relativement simple à appréhender et à mettre en œuvre.
Elle est très efficace pour détecter les attaques connues, dont la signature\index{signature d'attaque} est transcrite sous forme de règle servant à proscrire les comportements malicieux.
Ces règles sont faites de telle façon que le taux de faux positifs soit peu élevé.
De plus, cette méthode de détection est centrée sur l'observation du comportement individuel des nœuds voisins des sentinelles, et permet donc le plus souvent de localiser efficacement l'origine d'une attaque.
En revanche, elle souffre d'une mauvaise adaptation aux attaques nouvelles, dont le profil n'est pas connu, et risque ainsi de ne pas détecter d'intrusion alors même que les performances du réseau se retrouvent fortement dégradées.

        \paragraph{Respect des spécifications}
Une troisième classe de méthodes de détection regroupe les méthodes, moins fréquentes, basées sur le respect de spécifications réalisées en amont du déploiement du réseau.
La détection basée sur les anomalies\index{anomalie} détecte les écarts de performance, celle basée sur les signatures\index{signature d'attaque} relève les comportements suspects; cette troisième catégorie cherche à combiner les avantages de chacune, en se reposant sur des spécifications élaborées par l'opérateur ainsi que sur des contraintes qui caractérisent le bon fonctionnement du réseau.
Les écarts au profil normal spécifié permettent de détecter les attaques en identifiant leur effet, tandis que les contraintes fixées par ces spécifications entrainent la détection de comportement individuels malicieux~\cite{BMS13}.

Comme les spécifications sont réalisées par un opérateur humain (il ne s'agit pas d'un modèle généré par un algorithme), le taux de faux positifs est relativement faible.
En revanche, l'élaboration de ces spécifications peut s'avérer très complexe, et prendre beaucoup de temps.
Ces méthodes sont relativement peu implémentées dans la littérature.
%2}}}

    \subsubsection{Des architectures adaptées}
%{{{2
Nous allons donner ici quelques exemples de propositions concrètes de systèmes de détection d'intrusion, sans chercher à couvrir l'ensemble des nombreuses catégories présentées.
La plupart d'entre elles se basent sur l'emploi d'architectures (\idx{clusterisation}, architectures de sécurité\index{architecture de sécurité}) ou de protocoles (\idx{routage}) déjà existants, auxquelles elles rajoutent des systèmes dédiés à la détection des attaques.

\paragraph{Quelques \IDS pour réseaux clusterisés\index{clusterisation!réseau clusterisé}}
Le système SecCBSN (\textit{Secure Communications of Cluster-Based Sensor Network}, « communications sécurisées des \rcs clusterisés ») introduit une architecture de sécurité à laquelle est adjointe un \ids pour sécuriser au mieux possible les réseaux clusterisés~\cite{HHC07}.
Chaque nœud implémente trois modules: l'un intervient pour l'\election du \ch, réalisée sur le modèle de \leach.
Le deuxième module est chargé d'apporter \idx{authentification} et \idx{chiffrement} pour les communications entre les capteurs au sein du réseau, et utilise notamment une version retouchée du certificat TESLA.
\nomenclature{SecCBSN}{\textit{Secure Communications of Cluster-Based Sensor Network}}
Ajout notable par rapport aux architectures que nous avions vues jusque là, un troisième module module est chargé de la détection des nœuds compromis.
Lorsqu'il détecte des comportements anormaux de la part de nœuds donnés, ce module fait propager une alarme jusqu'à la \sdb pour l'informer de l'état malicieux de ces capteurs.
Ce module se base sur un ensemble assez complexe de règles permettant de vérifier à la fois le bon comportement (non compromission) et le bon fonctionnement (disponibilité) des nœuds voisins.
Le système dans son ensemble est finalement proche des mécanismes que nous utilisons, sauf que tous les capteurs réalisent une activité de surveillance.

Notre système en question repose sur les travaux de \textsc{Lai} et \textsc{Chen}, qui ont proposé en 2008 une solution proche mais dont seuls certains capteurs effectuent des opérations de surveillance~\cite{LC08}.
Ces nœuds de contrôle, appelés \textit{gNodes}\index{gNode@\textit{gNode}} (pour \textit{guarding nodes}) par leurs auteurs, sont sélectionnés au sein du cluster pour effectuer les opérations de surveillance, permettant ainsi aux autres nœuds d'économiser d'autant en énergie.
Lorsque le \ch reçoit plusieurs rapports de la part de \textit{gNodes} distincts (ceci afin de limiter les cas de faux positifs), le \CH en vient à considérer le capteur dénoncé comme étant corrompu et prend des mesures en conséquences.
Bien que les simulations menées fournissent de bons résultats en termes de préservation des ressources, l'étude ne se penche pas sur le processus utilisé pour la désignation des nœuds de contrôle, et ne traite pas de leur \idx{renouvellement} dans le temps, ce qui fait que les même capteurs supportent la tâche de surveillance sur de longues durées.

C'est sur ces points que se concentrent la majeure partie des travaux présentés dans cette thèse: les \cns que nous utilisons ne sont qu'une version renommée des \textit{gNodes}, pour lesquels a été proposé dans un premier temps un \idx{renouvellement} périodique~\cite{GMT12}.
Nous nous attachons, dans le choix des nœuds de surveillance, à répartir au mieux la consommation énergétique dans le cluster; d'autres études cherchent à désigner ces nœuds de manière à obtenir la meilleure couverture possible, et à maximiser les chances de détection des nœuds compromis à l'aide d'heuristiques pour le placement de l'\IDS~\cite{INK09}.

        \paragraph{Lutter contre les pertes de paquets}
À travers l'usage plus ou moins direct de systèmes de détection d'intrusion, plusieurs architectures déjà établies ont été adaptées dans le but de faciliter la détection de nœuds corrompus au sein du réseau.

\olsr (\textit{Optimized Link State Routing protocol}) est l'un des protocoles de \idx{routage} les plus souvent mis en place dans les \rcs.
Une nouvelle version de ce protocole, appelée DLSR (\textit{DDoS-preventing \olsr})~\cite{MKASF10}, a été proposée en vue de reprendre ses fonctionnalités tout en ajoutant des mécanismes de détection contre les attaques de \dds distribuées.
Lorsque la \sdb reçoit un trafic trop important, qui menace de saturer sa capacité de traitement ou bien qui crée de trop fortes congestions sur ses liens, elle envoie un signal dans tout le réseau en vue de prévenir les nœuds; le trafic est alors analysé, par chaque nœud relais, par un automate d'apprentissage afin de détecter et supprimer les paquets malveillants.
DLSR fait signaler par la \sdb la fin de l'attaque, afin que les nœuds cessent les opérations nécessaires au filtrage des paquets.
\nomenclature{DDoS}{\textit{Distributes Denial of Service}}
\nomenclature{DLSR}{\textit{DDoS-preventing \olsr}}

Basé sur des mécanismes simples, REWARD reprend la règle de retransmission pour assurer qu'aucun nœud ne commet d'attaque de type \textit{blackhole}\index{trou!trou noir}~\cite{Kar05}.
À chaque émission de paquet, le nœud émetteur (qu'il soit émetteur d'origine ou relais, sauf s'il est le dernier relais de la route empruntée) vérifie que le destinataire retransmet le paquet sur le saut suivant.
Cette procédure ne permet pas de détecter deux attaquants « alignés » sur la route des paquets, qui agiraient de concert, le premier transmettant le paquet au second, sans lever d'alerte lorsque celui-ci se contenterait de supprimer le paquet.
Aussi lorsque l'on suspecte que des attaquants coopèrent\index{coopération}, REWARD ne se contente plus de ne faire surveiller que ses voisins directs par un capteur, mais propose d'augmenter la puissance de transmission lors du transit du paquet, de sorte que la règle puisse être vérifiée par chaque nœud sur les deux sauts suivant son émission.

Une conséquence des attaques de type \textit{sinkhole}\index{trou!puits}, outre la perte des paquets, est la consommation supplémentaire en énergie que l'attaquant impose à ses voisins.
Une étude se base sur ce constat pour établir une méthode « géostatistique » de détection pour ces attaques~\cite{SKDM14}.
Le principe consiste à échantillonner périodiquement des zones du réseau, en demandant aux nœuds leur énergie résiduelle.
Un traitement sur ces données permet de déterminer les zones ou l'énergie des nœuds est plus faible qu'elle ne le devrait, et donc où une attaque de type \textit{sinkhole}\index{trou!puits} est peut-être en cours.
Ce traitement consiste en l'application d'un modèle de survie, plus précisément une régression de \textsc{Cox} (modèle à risque proportionnel), qui est réalisé soit par la \sdb, soit, dans une version adaptée, par des nœuds de \idx{confiance} au sein du réseau.

        \paragraph{Sécurité système\index{securité@sécurité!sécurité système}}
Outre les systèmes de détection d'intrusion machine, qui opèrent sur le système d'exploitation du nœud même sur lequel ils s'exécutent, il est possible de faire vérifier l'\integrite des systèmes par d'autres entités.
Il existe ainsi une proposition concernant l'usage d'agents mobiles\index{mobilité} dans le réseau, qui se déplacent de nœud en nœud pour vérifier que le code binaire de leur système n'a pas été altéré~\cite{HR13}.
Si des capteurs corrompus sont détectés et que leur état ne peut être immédiatement restauré (soit que l'opération soit impossible, soit que l'agent mobile\index{mobilité} reçoive l'alarme d'un autre capteur et n'ait pas encore eu le temps d'accéder au nœud affecté), ces agents permettent également de propager l'alerte dans le réseau.

        \paragraph{}
Plusieurs études se proposent de résumer les avancées des solutions proposées en termes de systèmes de détection d'intrusion~\cite{ME13,MS14}, et fournissent de nombreux autres exemples de ces systèmes.
Les travaux de synthèse récents de \textsc{Butun, Morgera} et \textsc{Sankar}, qui classent et détaillent les \IDS pour \manet et \rcsfs~\cite{BMS13} sont notamment très complets.
Et parce qu'aucun système \IDS n'est parfait, on trouve même des travaux réalisés sur les limites de certaines méthodes de détection et des mécanismes de \idx{confiance} (que nous verrons plus bas), assorties de mesures permettant d'améliorer l'efficacité de ces méthodes~\cite{CQW12}.
%2}}}
%1}}}
