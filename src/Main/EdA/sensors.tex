% vim: set spelllang=fr foldmethod=marker:
\section{Réseaux de capteurs sans fil}

\subsection{Architecture des réseaux}


In addition, WSNs have following specific security objects:
- Forward secrecy: preventing a node from decrypting any future secret messages after it
leaves the network
- Backward secrecy: preventing a joining node from decrypting any previously transmitted
secret message
- Survivability: providing a certain level of service in the presence of failures and/or
attacks
- Freshness: ensuring that the data is recent and no adversary can replay old messages
- Scalability: supporting a great number of nodes
- Efficiency: storage, processing and communication limitations on sensor nodes must be
considered
(2008 Li Gong)





\subsection{Algorithmes de clusterisation}

\subsection{Gestion des ressources}

Power conservation and security enhancement in WSNs - a priority based approach (2014 Sivakumar Amirthavalli Senthil)

K. Piotrowski, P. Langendoerfer, and S. Peter. How Public Key Cryptography Influences Wireless Sensor Node Lifetime. In Proc. of the 4th ACM SASN, 2006.
PLP06

Y. Law, J. Doumen, and P. Hartel. Survey and benchmark of block ciphers for wireless sensor networks. Technical Report TR-CTIT-04-07, Centre for Telematics and Information Technology, University of Twente, The Nether- lands, 2004.

\subsection{Sureté et résilience}\label{ea:ssec:safety}

La sureté et la résilience des réseaux, traités en \ssref{ea:ssec:safety}, ne sont pas des problématiques de sécurité (car aucune attaque volontaire ne rentre en compte), mais il s'agit de problématiques assez proches de la disponibilité des réseaux.


