% vim: set spelllang=fr foldmethod=marker:
\section{Théorie des jeux}

The authors of~\cite{MT08} categorize game-theoretic approaches in WSNs into three main categories: energy efficiency, security, and pursuit-evasion.
Pursuit-evasion games consist in a set of mobile players trying to “capture” another set, to optimize tracking, while the opponents aim at avoiding detection.
The other two categories are rather explicit: when looking for energy efficiency, games are used to save as much energy as possible, by optimizing either the network topology or their own behavior~\cite{CPF09}.
Security games, of course, oppose normal sensors to attackers from inside or outside the network.
In a more recent survey~\cite{SWKC12} the pursuit-evasions games are just part of a broader “application” category (along with data collection for instance), whereas energy efficiency has been split into network management (resources, power) and communication (QoS, topology, routing design).

In~\cite{MMZ09} the authors model the interactions between the nodes and an IDS as a Bayesian game (\ie with partial information: the IDS does not know \apriori whether a given node is compromised).
They analyze the Nash equilibrium of this game to design a secure routing protocol.
Bayesian games are also used in~\cite{Ham12}, but they are invoked \aposteriori, as a verification tool rather than a way to design the solution.

Repeated games are models involving sequences of history-dependent game strategies: the players perform a sequence of actions, and their strategies is influenced by what the other players have done in the past.
Those games are used in~\cite{AD07} to set up an IDS, or in a less generic solution in~\cite{Red09}, which relies on the acknowledgments upon transmissions to detect a malicious node located in the forward data path.

Our approach differs as the potential death (by exhaustion) of nodes is included in the game constraints, alongside payoff values.
To the best of our knowledge, works related to this setting consider all dimensions either as energy or as payoff, and deal only with conjunction~\cite{chatterjee12,velner12a}.
In this case, the algorithms follow the same structure: the game objective is decomposed according to each dimension and finite-memory winning strategies for each dimension are retrieved.
Then these strategies are combined, possibly yielding an infinite-memory strategy.

Some other works~\cite{velner12b} deal with more involved objectives based on combination of mean-payoff objectives using sum, $\max$, and $\min$ operators.

\smallskip

A little bit further from our approach, some authors consider games with both a payoff requirement and a parity objective~\cite{chatterjee05}.
In that case the parity objective ensures that the system behaves correctly, while the payoff represents a quantitative goal.
In this case, winning strategies may require infinite memory, although an approximation can be obtained with finite memory.
This kind of game is however ill suited to the modeling of \wsn, since here the energy is an important factor to the life of the system.





% ======== >>> §3 <<< WAS NOT MODIFIED, IS NEW
Elements from game theory have been used in several studies to detect DoS attacks in WSNs.
In
\cite{MMZ09},
\textit{Mohi et al}. propose another another way to secure the LEACH protocol against selfish behaviors.
With S-LEACH, the BS uses a global Intrusion Detection System (IDS) while LEACH CHs implement local IDSs.
The interactions between nodes are modeled as a Bayesian game, that is, a game in which at least one player (here, the BS) has incomplete information about the other player(s) (in this case: whether the sensors have been compromised or not).
Each node has a ``reputation'' score.
Selfish nodes can cooperate (so as to avoid detection) or drop packets.
The authors show that this game has two Bayesian Nash equilibriums which provide a way to detect selfish nodes, or to force them to cooperate to avoid detection.




% A Bayesian Game Approach for Preventing DoS Attacks in WSNs
Elements from game theory have been used in several studies to detect DoS attacks in WSNs.
In~\cite{MMZ09}, the authors propose another another way to secure the \leach protocol against selfish behaviors.
With S-\leach, the BS uses a global Intrusion Detection System (IDS) while \leach CHs implement local IDSs.
The interactions between nodes are modeled as a Bayesian game, that is, a game in which at least one player (here, the BS) has incomplete information about the other player(s) (in this case: whether the sensors have been compromised or not).
Each node has a ``reputation'' score.
Selfish nodes can cooperate (so as to avoid detection) or drop packets.
The authors show that this game has two Bayesian Nash equilibriums which provide a way to detect selfish nodes, or to force them to cooperate to avoid detection.
