% vim: set spelllang=fr foldmethod=marker:
\subsection{Discussion: plausibilité des attaques}

By ``plausible'' attacks we mean \dos attacks that a real attacker would be likely to launch on a physical \wsn.
On the contrary, many attacks have been described in the literature which exploit some flaws of ``obscure'' protocols, need a lot of retro-engineering/protocol analysis, sometimes require a lot of expensive materials, and/or are not likely to succeed.
Attackers gifted with common-sense will probably use simpler/cheaper/more efficient methods.
Question is: what are those methods?

We will first list the existing (in literature, at least) \dos attacks.
Then to attempt to bring an answer to the previous question, we will try to bring out attackers' motivations for their acts, and then to categorize the attacks according to their suitability to address those motivations.

\subsubsection{Motivations for the attackers}

We are assuming that the reasons pushing the attackers into setting up a \dos attack may be the following.

\paragraph{Disabling communication}
The attacker wants to bring the network down.
No communication between the nodes.
No detection of the events from the covered area must be forwarded to the \bs.

\paragraph{Destroying the network}
The existence of the network is annoying for the attacker.
They want to brig it down permanently, maybe because that they can not afford to bring it down regularly for one-time needs.

\paragraph{Retrieving data}
The attacker wants to retrieve data from the network.
This is not generally related to \dos attacks, but some may help.

\paragraph{Tampering results}
The attackers want their target to receive false data (\eg in movies: to show an empty room on security cameras whereas there is someone sneaking inside).

\subsubsection{Relevant attacks}

\paragraph{Disabling communication}
The simplest \dos attack against a wireless network consists in jamming its signal.
This is very easy to set up, even if it may be expensive in materials.
For those reasons, it is probably one of the most frequent attack led by military forces.

Jamming and its derivatives (reactive jamming, random jamming, deceptive jamming, intelligent jamming, cyber-mines\dots) cover simple techniques on the physical layer as well as more advanced mechanisms on data link layer (producing collision against given packets \etc).
The simple physical jamming of the channel seems to be the most probable attack, as it needs absolutely no retro-engineering of the target network (the attackers only need to know on which frequency and with which strength they have to emit).

It also renders quite useless many network layer \dos attacks such as loops in the network or ACK spoofing.
Higher layers attacks such as de-synchronisation or flooding are in the same case.
Sybil attacks were originally led to bypass security mechanisms.
While good at it, they do not bring particular benefit outside this particular case.
So all those are attacks whose sole purpose is to bring the service down.
But jamming is simpler.

\paragraph{Destroying the network}
Depending on the area where the network is deployed, physical destruction of the sensors may occur.
But no protocol can handle that, so this is out of the study.
Another way to permanently disable the network without breaking the nodes is energy exhaustion.
If run efficiently, it can fasten the consumption of the energy resources available for the nodes.
Led by a few compromised nodes, it could cost less than jamming over a long period.
But it requires a good knowledge of the deployed MAC protocol in the WSN.

\paragraph{Retrieving data}
Strictly speaking retrieving data does not fall within \dos attacks.
But some network attack consisting in announcing false short paths to the \bs can enable a compromised node to gather much data from a large part of the network.
Black hole attacks proceed like this, and then consist in not forwarding packets.
If data is gathered by the attacker for external process, this implies that the compromised node must be able to forward collected messages outside the network (which concerns confidentiality and is outside the scope of the study).

Spoofed routing information and then black hole attacks look like the more efficient attacks.
Grey hole, on-off hole attacks are just more complex for no more benefit.
Sink hole attacks harm the routing of messages; but once again, jamming is simpler.

\paragraph{Tampering results}
Data integrity may be closely related to \dos attacks when tempering the payload of the packets renders useless the service provided by the network.
An attacker could lie and send false data, too much or too little data (to sidestep the collected results), or tamper the packets it forwards.


\subsubsection{Conclusion}

Most plausible attacks seem to be:
\begin{itemize}
	\item jamming attacks;
	\item some routing attacks, and especially black hole attack;
	\item energy depletion (but requires good knowledge of protocols in use);
	\item false/tampering data.
\end{itemize}

Most jamming derivatives outside reactive jamming maybe need tests and adjustments.
Other attacks whose sole consequence is to bring the network down are generally much more complex to implement.
Most MAC layer level attacks require a good knowledge of the protocols deployed.
Unfairness and greediness are not very likely to motivate an attack in a \wsn.
