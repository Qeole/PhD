% vim: set spelllang=fr foldmethod=marker:
\section{Déni de service}\label{ea:sec:dos}

Une attaque dite de « déni de service » menée dans un réseau informatique est une attaque réalisée dans le but de nuire au fonctionnement normal de ce réseau.
Il existe de très nombreuses façons de procéder, et on recense par conséquent une multitude d'attaques par déni de service existantes.
L'état de l'art dans ce domaine (et à propos de la sécurité en général, par ailleurs) a ceci de particulier qu'il comporte deux points de vue: celui de l'attaquant et celui du « défenseur ».
Il est indispensable de pouvoir définir le modèle d'une attaque pour pouvoir proposer des contre-mesures adéquates.
Et de façon plus ou moins réciproque, les mécanismes de protection mis en place au fil du temps poussent les attaquants (ou les chercheurs) à développer de nouvelles attaques pour les contourner.

Les réseaux de capteurs se retrouvent malheureusement très exposés aux attaques de déni de service~\cite{RM11}, du fait de:
\begin{itemize}
    \item leurs ressources extrêmement limitées, et principalement en termes d'énergie;
    \item leurs faibles capacités, qui peuvent introduire des délais (latence dans les communications ou délai de traitement)
    \item leur exposition aux attaques physiques;
    \item la faible fiabilité du medium de transmission, en termes de confidentialité ou de collisions
    \item leur gestion réalisée à distance;
    \item l'absence d'une gestion centralisée (et l'impossibilité de connaître avec précision le statut des autres nœuds);
\end{itemize}

Dans cette section nous présenterons d'abord les principales attaques répertoriées dans les réseaux de capteurs, puis nous aborderons les mécanismes proposés en réponse dans la littérature.
Auparavant, nous allons voir qu'il existe plusieurs façons de classer ces attaques.

\subsection{Différentes classifications}

\subsubsection{Selon l'objectif recherché}
Une première façon de classer les attaques est de considérer l'objectif de l'attaquant.
Les buts principaux de ces attaques sont:
\begin{itemize}
    \item l'accaparement de ressources pour les besoins propres de l'attaquant, au détriment des autres agents du réseau (par exemple, monopolisation du canal de transmission pour l'envoi des données de l'attaquant exclusivement). Il s'agit de comportements « cupides » (\textit{greedy} en anglais);
    \item la réduction, voir l'annihilation de la capacité du réseau à assurer correctement les services pour lequel il a été déployé, afin de nuire aux exploitants de ce réseau. Cette nuisance peut s'exprimer par des pertes financières, par exemple lorsqu'une telle attaque est menée sur Internet contre un site de commerce en ligne; dans le cas des réseaux de capteurs sans fils, et notamment lorsqu'ils sont utilisés dans un cadre militaire, l'exploitant d'un réseau peut alors se voir priver de l'accès aux informations stratégiques que devaient récolter les capteurs. On parle le plus souvent d'attaques de type \textit{jamming} en anglais, qui se traduit selon le cas par « brouillage », « encombrement », mais d'autres attaques reposant sur la privation de sommeil des capteurs, ou bien leur destruction physique, peuvent aussi être réalisées dans cet objectif;
    \item plus rarement, l'induction en erreur de l'exploitant du réseau de capteurs. Pour ceci l'attaquant cherche à fausser les résultats collectés (changement d'environnement des capteurs) ou transmis (altération du contenu ou du volume de données transmises).
\end{itemize}

\subsubsection{Selon la situation de l'attaquant}
On distingue également les attaques selon la provenance de l'attaquant, à savoir s'il fait partie du réseau, et mène par exemple une attaque sur le protocole de routage employé, ou bien s'il agit depuis l'extérieur du réseau, depuis une machine qui n'est pas reconnue comme faisant partie du réseau de capteurs.
Ce deuxième cas peut être illustré par un attaquant qui chercherait à brouiller de manière globale toutes les fréquences radio utilisées par les capteurs pour communiquer.

Un attaquant extérieur n'a pas forcément connaissance de la façon dont fonctionne le réseau (architecture, protocoles employés, mesure de détection mises en place).
Il peut mener une attaque sans ces informations (cas du brouillage des fréquences par exemple), ou bien justement chercher à s'introduire dans le réseau.
Cette intrusion peut être réalisée par l'attaquant soit en faisant accepter l'un de ses appareils propre aux autres agents du réseau, soit en compromettant l'un des agents jusqu'alors légitime dans le réseau.
Une fois l'accès interne obtenu, il devient généralement beaucoup plus facile d'accéder à des informations sur le fonctionnement du réseau.
L'attaquant, en fonction des mesures mises en place, devient aussi susceptibles d'être détecté et exclu du réseau.
Les attaques menées depuis l'intérieur jouent souvent sur les paramètres des protocoles employés, et sont souvent plus « subtiles », plus délicates à détecter et identifier pour l'opérateur du réseau si aucune méthode de détection d'intrusion n'a été mise en place.

\subsubsection{Selon les capacités de l'attaquant}
On peut encore distinguer les attaques selon la puissance dont dispose l'attaquant, que ce soit en terme de calcul, d'émission électromagnétiques, ou bien d'alimentation en énergie.
Un attaquant peut n'avoir à disposition qu'un capteur normal (\textit{mote-class attacks} en anglais), qu'il lui appartienne ou bien qu'il ait été compromis parmi les capteurs déployés à l'origine.
Il se retrouve dans ce cas avec des machines similaires aux appareils attaqués, mais n'a pas besoin de rester sur place et peut mener des attaques sur la durée.
Il peut également être équipé d'un ordinateur plus puissant (\textit{laptop-class attacks}), voir de matériel militaire spécialisé.
Dans ce cas, il est plus aisé de déployer de la puissance (puisque l'attaquant peut s'affranchir des limites imposées par les batteries des capteurs) et par exemple de brouiller en continu toute une plage de fréquence.
Il faut néanmoins que le matériel utilisé reste déployé le temps de mener l'attaque, ce qui peut s'avérer coûteux sur la durée.

\subsubsection{Attaques actives, passives}
Cette méthode de classement ne s'applique pas ici.
Les attaques dites « passives » n'interfèrent pas avec le fonctionnement normal du réseau: il peut s'agir par exemple d'écoute clandestine en vue de collecter des données (atteinte à la confidentialité des communications), mais par définition les attaques de déni de service sont des attaques dites « actives », au cours desquelles l'attaquant introduit de nouveaux comportement dans le réseau.

\subsubsection{Selon le paradigme considéré}\label{ea:sss:paradigm}
Nous abordons à présent des méthodes effectives de classement, qui permettent de catégoriser plus précisément les attaques (plutôt que d'établir une simple dissociation des catégories sur un nombre restreint de critères).

Certaines études se penchent sur une classe d'opérations réalisées par les réseaux de capteurs, et répertorient les attaques et les contre-mesures qui s'appliquent à cette classe~\cite{JPD06,OX09}.
Les paradigmes suivants peuvent être considérés:
\begin{itemize}
    \item la collecte et la transmission simples des données (paradigme qui se concentre sur une collecte simple et un envoi direct à la station de base, sans traitement, sans routage dans le réseau);
    \item la transmission des données d'un point à un autre du réseau (\cad tout ce qui touche au routage des paquets dans le réseau, donc en faisant intervenir la retransmission des paquets par plusieurs nœuds intermédiaires);
    \item la réception et le traitement de commandes (dans le cas où les nœuds sont susceptibles de communiquer entre eux et de s'échanger des commandes, pouvant mener à des changement de configuration des capteurs);
    \item l'organisation autonome du réseau (les réseaux de capteurs s'organisent de façon autonome, mais des attaques peuvent chercher à interférer avec la formation d'une architecture cohérente; dans ce paradigme sont inclus les protocoles de clusterisation éventuellement mis en application);
    \item l'agrégation de données (qui consiste à agréger, éventuellement compresser les données reçues avant retransmission, dans le but de limiter la taille et le nombre de paquets envoyés, afin de minimiser l'utilisation du canal de transmission, et surtout la consommation en énergie des capteurs);
    \item l'optimisation du modèle utilisé (qui intervient lorsque les capteurs ont des décisions à prendre, basées sur le contenu des paquets, telles que la retransmission directe ou différée, le niveau de sécurité à fournir\dots).
\end{itemize}
Cette méthode de classement est la plus utile pour se concentrer sur un point particulier du fonctionnement d'un réseau, et relever toutes les failles susceptibles d'affecter les opérations concernées.

La \tabref{ea:tab:paradigm} située plus bas présente un classement selon ces paradigmes des attaques que nous allons évoquer.

\subsubsection{Selon les couches de protocoles concernées}
Une seconde méthode efficace de classement consiste à procéder par couches de protocoles, en se basant sur le modèle \textsc{Tcp/Ip} (schéma concret lui-même issu du modèle théorique \textsc{Osi}, de l'anglais \textit{Open Systems Interconnection}).
Ce modèle est rappelé en \figref{ea:fig:tcpip}.
\begin{table}[!ht]
    \centering
    \begin{tabular}{c |c| l}
        \multicolumn{2}{c}{} & Exemple:\\
        \cline{2-2}
        5 & Application & \textsc{Http, Ftp, Ssh}\\
        \cline{2-2}
        4 & Transport & \textsc{Tcp, Udp}\\
        \cline{2-2}
        3 & Réseau & \textsc{Ip}\\
        \cline{2-2}
        2 & Liaison & \textsc{Ieee 802.11 (Cdma/Ca)}\\
        \cline{2-2}
        1 & Physique & ondes électromagnétiques\\
        \cline{2-2}
     \end{tabular}
    \medskip
    \caption{Modèle \textsc{Tcp/Ip}}\label{ea:fig:tcpip}
\end{table}

Ce classement permet une revue efficace, couche par couche, de la plupart des attaques connues.
C'est donc selon ce critère que nous allons maintenant présenter les principales attaques de déni de service connues dans les réseaux de capteurs.

\subsection{Différents types d'attaques}

    \subsubsection{Couche physique}
La couche physique du réseau correspond au medium physique employé pour la transmission des données entre deux nœuds, et à la façon dont le signal est transmis au travers de ce medium.
Dans le cas de réseaux sans fil, le signal est propagé sous forme d'ondes électromagnétiques qui se déplacent dans le vide (ou bien, sans en être affectées, à travers l'atmosphère).
Sauf s'il est fait usage d'une antenne directionnelle pour l'émission, ces ondes sont envoyées dans toutes les directions, et tout appareil à portée équipé d'un récepteur se retrouve donc en mesure de recevoir les paquets émis.

        \paragraph{Brouillage de fréquences}
Le brouillage de fréquences (\textit{frequency jamming} en anglais) consiste pour l'attaquant à émettre un signal parasite, un « bruit » électromagnétique, sur les fréquences concernées, de façon à ce que la cible visée ne puisse plus recevoir de façon correcte les paquets qui lui sont envoyés par les nœuds légitimes.
Le brouillage peut être réalisé à l'aide d'une antenne directionnelle pour viser une cible en particulier; mais dans le cas d'un réseau de capteurs, l'attaquant cherche en général à émettre un bruit dans toutes les directions afin d'affecter le plus grand nombre de nœuds possible.
L'attaque peut être menée de façon sporadique, de manière à produire un déni de service partiel, ou bien en continu.
Si la ou les machine(s) émettant le signal parasite possède(nt) une portée suffisante pour couvrir toute l'étendue géographique du réseau, l'intégralité des capteurs peuvent se retrouver dans l'impossibilité d'utiliser les fréquences brouillées.
Si de plus, le brouillage est mené sur toute la plage de fréquences accessibles aux capteurs, les communications deviennent totalement impossibles à établir dans le réseau.\\
Il est à noter que mener une telle attaque peut être coûteux en matériel pour l'attaquant, surtout s'il vise plusieurs fréquences et/ou un brouillage continu dans le temps.
Typiquement, un capteur corrompu ne sera pas en mesure de mener cette attaque sans épuiser très rapidement sa batterie.

    \subsubsection{Couche liaison de données}
La couche de liaison de données fournit les moyens fonctionnels et procéduraux pour le transfert des données entre deux entités du réseau.
Elle permet aussi, le plus souvent, de détecter et éventuellement corriger certaines erreurs survenues sur la couche physique (en cas de perturbation ou dégradation du signal électromagnétique).
Elle se décompose en deux sous-couches: la couche de contrôle de la liaison logique (\textsc{Llc}, pour \textit{Logical Link Control} en anglais) et la couche du contrôle d'accès au support (\textsc{Mac}, pour \textit{Media Access Control}).
C'est principalement cette seconde couche qui nous intéresse ici: le protocole de couche \textsc{Mac} définit la manière dont les différents agents du réseau accèdent au medium de transmission de façon à limiter les collisions, et à garantir un accès le plus souvent équivalent au medium pour tous les nœuds.
Les différents modes d'accès au medium existants sont résumés dans la \tabref{ea:tab:mac}; certains consistent à créer des canaux de transmission distincts, tandis que d'autres déterminent l'accès à une même bande de fréquence en instaurant des règles.
Ces règles peuvent être contournées, et la couche \textsc{Mac} va donc se retrouver associée à plusieurs types d'attaques.

\begin{table}[!ht]
    \caption{Méthodes d'accès au medium de transmission}\label{ea:tab:mac}
    \centering
    \medskip
    \begin{small}
        \begin{tabular}{m{.2\textwidth}|m{.2\textwidth}|m{.48\textwidth}}
            \toprule
            \textbf{Nom anglais} & \textbf{Traduction} & \textbf{Description}\\
            \midrule
            \multicolumn{3}{c}{Commutation de circuits et création de canaux}\\
            \midrule
            \textit{Frequency Division Multiple Access} (\textsc{Fdma}) & Accès multiple par répartition en fréquence & Plusieurs canaux basés sur des fréquences différentes\\
            \midrule
            \textit{Code division multiple access} (\textsc{Cdma}) & Accès multiple par répartition en code & Étalement du spectre de fréquence utilisé en conjonction avec techniques comme les sauts de fréquences ou la génération de bruit pseudo-aléatoires (avec la même séquence pseudo-aléatoire appliquée au signal côté émetteur comme côté destinataire)\\
            \midrule
            \textit{Time division multiple access} (\textsc{Tdma}) & Accès multiple à répartition dans le temps & Un seul canal dont l'accès est réparti par créneaux dans le temps\\
            \midrule
            \textit{Space division multiple access} (\textsc{Sdma}) & Accès multiple à répartition dans l'espace & Plusieurs canaux spatiaux obtenus à l'aide d'antennes directionnelles. À noter: les antennes directionnelles augmentent sensiblement le coût de production des capteurs.\\
            \midrule
            \multicolumn{3}{c}{Mode d'accès par paquet}\\
            \midrule
            \textit{Contention based random multiple access methods} & Accès par contention & Contention par le nœud du paquet à envoyer jusqu'à ce que le protocole le lui autorise. Dans cette catégorie se trouve notamment le protocole \textsc{Csma/Ca} (\textit{Carrier Sense Multiple Access with Collision Avoidance}, accès multiple par écoute du canal avec esquive de collision), très utilisé dans les réseaux sans fil (\textsc{Ieee~802.11} entre autres).\\
            \midrule
            \textit{Resource reservation (scheduled) packet-mode protocols} & Réservation des ressources & Réservation par un nœud des ressources (par exemple: créneau temporel) nécessaires à la transmission)\\
            \midrule
            \multicolumn{3}{p{.95\textwidth}}{D'autres modes d'accès par paquet (\textit{token passing, polling}) existent mais ne sont pas utilisés dans les réseaux de capteurs}\\
            \bottomrule
         \end{tabular}
     \end{small}
\end{table}

        \paragraph{Création de collisions et brouillage « intelligent »}
Lorsque plusieurs nœuds dont les portées se chevauchent émettent de façon simultanée en utilisant la même fréquence (donc sur un même canal), il se produit des collisions.
La plupart des protocoles \textsc{Mac} employés avec les réseaux de capteurs introduisent dans les trames un champ contenant une somme de contrôle, qui permet de vérifier l'intégrité de la trame.
Mais cette somme de contrôle ne permet pas, la plupart du temps, de corriger d'éventuelles erreurs (aucun des standards \textsc{Ieee} 802.11 (Wi-Fi), 802.15.1 (Bluetooth) ou 802.15.4 (ZigBee, 6LoWPAN) n'inclue de code de correction des erreurs).
Si un seul bit de la trame est altéré, celle-ci est donc rejetée par le destinataire.
Un attaquant peut donc chercher à produire des collisions en émettant un signal en même temps qu'un nœud légitime, afin que le destinataire ne puisse pas recevoir correctement la trame qui lui est destinée.
Ce principe de collision est identique au brouillage mené sur la couche physique; mais lorsque l'attaquant a connaissance du protocole de couche \textsc{Mac} employé, il lui est possible d'affiner son attaque, et de remplacer un brouillage « brut », continu et onéreux, par un brouillage « intelligent ».
Il existe plusieurs façons de procéder~\cite{PI11}:
\begin{itemize}
    \item la première, comme vu plus haut, n'est pas « intelligente » et consiste à brouiller les fréquences de manière continue, en produisant un bruit aléatoire;
    \item une technique plus difficile à détecter consiste à envoyer des paquets normaux mais de façon continue, de façon à occuper le canal sans interruption. Elle empêche les nœuds légitimes de transmettre dans le cas ou le protocole de couche \textsc{Mac} est \textsc{Csma/Ca} par exemple;
    \item une technique plus économe consiste à envoyer des bits à intervalles distincts (réguliers ou non), dans l'espoir de créer des collisions. Il n'est alors plus nécessaire d'émettre en continu, et l'attaque est moins facile à détecter. Elle est bien sûr moins efficace, car l'attaquant n'a plus la garantie de brouiller tous les paquets émis;
    \item les attaques « intelligentes » à proprement parler reposent sur la connaissance des spécifications du protocole \textsc{Mac} employé, et consistent à jouer sur la nature des paquets de contrôle et les délais d'attente. Ainsi, pour le protocole \textsc{Csma/Ca} employé avec la suite \textsc{Ieee~802.11}~\cite{ieee802.11}, une trame spécifique dite \textsc{Rts} (\textit{Request To Send} en anglais) de demande de réservation du canal est suivie d'une trame \textsc{Cts} (\textit{Clear To Send}) après une durée \textsc{Sifs} (\textit{Short InterFrame Space}) si le destinataire est près à recevoir des données. En créant une collision sur la seule trame \textsc{Cts}, l'attaquant prévient entièrement l'échange de données utiles entre les nœuds.
D'autres trames peuvent être utilisées dans le même but, leurs acronymes sont dans la \tabref{ea:tab:smartjam}.
\end{itemize}

\begin{table}[!ht]
    \caption{Brouillage « intelligent »: corruption des trames de contrôle}\label{ea:tab:smartjam}
    \centering
    \medskip
    \begin{tabular}{c c c}
        \toprule
        Paquet analysé & Temps d'attente & Paquet à corrompre\\
        \midrule
        \textsc{Rts} & \textsc{Sifs} & \textsc{Cts}\\
        \textsc{Data} & \textsc{Sifs} & \textsc{Ack}\\
        \textsc{Rts} & \textsc{Difs} & \textsc{Data}\\
        Quelconque & \textsc{Difs} & \textsc{Rts} ou \textsc{Data}\\
        \bottomrule
    \end{tabular}
\end{table}

        \paragraph{Épuisement de la batterie}
        \paragraph{Accaparement du canal de transmission}
        \paragraph{Falsification d'accusés de réception}

    \subsubsection{Couche réseau}
        \paragraph{Falsification des informations de routage (accaparement du canal, création de boucles, \etc)}
        \paragraph{Retransmission sélective de paquets (trous gris, attaques « on/off »)}
        \paragraph{Puits}
        \paragraph{Trou noir}
        \paragraph{Trou de ver}
        \paragraph{Déluge de paquets « hello »}
        \paragraph{Attaque Sybil}
        \paragraph{Altération des données}

    \subsubsection{Couche transport}
        \paragraph{Déluge \textsc{Tcp} (ou équivalents)}
        \paragraph{Désynchronisation \textsc{Tcp} (ou équivalents)}

    \subsubsection{Couche application}
        \paragraph{Données erronées}
        \paragraph{Déluge de paquets}
        \paragraph{Désynchronisation}

    \subsubsection{Hors modèle}
        \paragraph{Destruction physique des capteurs}
        \paragraph{Altération des mesures}
        \paragraph{Attaques portées sur l'organisation du réseau (par exemple lors de l'application d'un protocole de clusterisation)}


\begin{table}[!ht]
    \newcounter{LayerNumber}
    \setcounter{LayerNumber}{1}
    \newcommand\num[1]{\theLayerNumber.~#1\stepcounter{LayerNumber}}
    \caption{Classement des attaques par couche du modèle \textsc{Tcp/Ip}}\label{ea:tab:layer}
    \centering
    \medskip
    \begin{small}
        \begin{tabular}{m{.21\textwidth}|p{.71\textwidth}}
            \toprule
            \multirow{3}{*}{Hors modèle}%
                & \num{Destruction physique des capteurs}\\
                & \num{Altération des mesures}\\
                & \num{Attaques portées sur l'organisation du réseau (par exemple lors de l'application d'un protocole de clusterisation)}\\
            \midrule
            \multirow{1}{*}{Couche physique}%
                & \num{Brouillage}\\
            \midrule
            \multirow{4}{*}{\parbox{.2\textwidth}{Couche liaison de données}}%
                & \num{Création de collisions, brouillage « intelligent »}\\
                & \num{Épuisement de la batterie}\\
                & \num{Accaparement du canal de transmission}\\
                & \num{Falsification d'accusés de réception}\\
            \midrule
            \multirow{8}{*}{Couche réseau}%
                & \num{Falsification des informations de routage (accaparement du canal, création de boucles, \etc)}\\
                & \num{Retransmission sélective de paquets (trous gris, attaques « on/off »)}\\
                & \num{Puits}\\
                & \num{Trou noir}\\
                & \num{Trou de ver}\\
                & \num{Déluge de paquets « hello »}\\
                & \num{Attaque Sybil}\\
                & \num{Altération des données}\\
            \midrule
            \multirow{2}{*}{Couche transport}%
                & \num{Déluge \textsc{Tcp} (ou équivalents)}\\
                & \num{Désynchronisation \textsc{Tcp} (ou équivalents)}\\
            \midrule
            \multirow{3}{*}{Couche application}%
                & \num{Données erronées}\\
                & \num{Déluge de paquets}\\
                & \num{Désynchronisation}\\
            \bottomrule
        \end{tabular}
    \end{small}
\end{table}

\begin{table}[!ht]
    \caption{Classement des attaques par paradigme (voir \sssref{ea:sss:paradigm})}\label{ea:tab:paradigm}
    \centering
    \medskip
    \begin{small}
        \begin{tabular}{m{.3\textwidth}|m{.62\textwidth}}
            \toprule
            Paradigme & Vulnérabilités \\
            \midrule
            Collecte et transmission simples & Jamming, collision induction, spoofing through spurious broadcasts, loss of confidentiality, physical tampering\\
            Routage des données dans le réseau & Simple attacks against routing: Selective forwarding, Black hole, Resource exhaustion, Data Corruption\\
            Réception et traitement de commandes & All of the above and Spoofing attack by an adversary issuing spurious commands\\
            Organisation autonome du réseau & All of the above and against routing protocol viz. Induced routing loops, Sink hole, Wormhole, Hello flooding\\
            Agrégation de données & All of the above and particularly vulnerable to Replay attack\\
            Optimisation des modèles & All of the above\\
            \bottomrule
         \end{tabular}
     \end{small}
\end{table}


attaques sur protocoles de clusterisation
destruction physique des capteurs



\subsection{Différents mécanismes de protection}

\subsubsection{Détection}
\subsubsection{Réaction}
\subsubsection{Prévention}

Indeed there are many existing attacks able to compromise the good working of a \wsn.%ONDEMAND \cite{ZJ09}.
Several mechanisms have been proposed to detect it and to provide countermeasures\cite{SSS11}. %ONDEMAND ,SSF13}.
Many consist in the implementation of trust based mechanisms\cite{MC10,F-GRL07} with agents applying set of rules\cite{RKKK13} on traffic to attribute a trust value to each of the nodes in the network.
In particular, Lai and Chen\cite{LC08} proposed to elect control nodes to monitor the traffic in clustered networks and to detect and react to \dos attacks.



In a first attempt to bring load balancing to this solution, we propose in other papers\cite{GMT12,BMM13} to reiterate the election periodically.
Simulations show a better load repartition among the cluster, but at this time our focus was not on designing an energy efficient election process for the \cns.

%%%%%%%%%%%%%%%%%%%================================================
% Detecting Denial of Service Attacks in Sensor Networks
%\subsection{[LC08] -- Inchangé}

%\subsubsection{Ancienne version}
%A cluster-based intrusion detection system is proposed in
%\cite{LC08}.
%It prevents sensor networks from DoS attacks.
%This solution deploys a set of special nodes called ``guarding nodes'' (gNodes) which observe, analyze the network traffic and report DoS attacks to their cluster head if an abnormal event happens.
%In each cluster there are three types of nodes, gNode, cluster head and sensor node.
%Any kind of nodes may be compromised.
%In this study, the detection approach for diverse attack types and the actions taken after detection are explored for the different node types.

%\subsubsection{Version Camera-Ready}
In
\cite{LC08},
the authors propose a system detection based on static election of a set of special nodes called ``guarding nodes'' which analyze the network traffic.
When detecting abnormal traffic from a given node, ``guarding nodes'' identify it as a compromised node and they inform the cluster head of it.
In this study, the authors show the benefit of their method by presenting numerical analysis of detection rate but they don't consider the energy of the elected node which dies very quickly.


% ======== >>> §3 <<<
%\subsection{[SM10] -- Ajouté}

% An adaptive learning routing protocol for the prevention of DDoS attacks in wireless mesh networks
In
\cite{MKASF10},
\textit{Misra et al}. propose a revised version of the OLSR protocol.
This routing protocol called DLSR aims at detecting distributed denial of service (DDoS) attacks and at dropping malicious requests before they can saturate a server's capacity to answer.
To that end, the authors introduce two alert thresholds regarding this server's service capacity.
They also introduce the use of Learning Automata (LAs), automatic systems whose choice of next action depends on the result of its previous action.
There is no indication in their work about the overhead or the energy load resulting from the use of the DLSR protocol.

% An adaptive learning routing protocol for the prevention of DDoS attacks in wireless mesh networks
In~\cite{MKASF10}, a revised version of the OLSR protocol is proposed.
This routing protocol called DLSR aims at detecting distributed denial of service (DDoS) attacks and at dropping malicious requests before they can saturate a server's capacity to answer.
To that end, the authors introduce two alert thresholds regarding this server's service capacity.
The authors also use Learning Automata (LAs), automatic systems whose choice of next action depends on the result of its previous action.
There is no indication in their work about the overhead or the energy load resulting from the use of the DLSR protocol.


% ======== >>> §4 <<<
%\subsection{[HI09] -- Modifié}

%% Optimal Sensor Placement for Detection against Distributed Denial of Service Attacks
The best way to detect for sure a DoS attack in a WSN is simply to run a detection mechanism on each single sensor.
Of course, this solution is not feasible in a network with constraints.
Instead of fitting out each sensor with such mechanism, \textit{Islam et al}. propose in
\cite{INK09}
to resort to heuristics in order to set a few nodes equipped with detection systems at critical spots in the network topology.
This optimized placement enables distributed detection of DoS attacks as well as reducing costs and processing overheads, since the number of required detectors is minimized.
But those few selected nodes are likely to run out of battery power much faster than normal nodes.

%% OLD VERSION
%The running of a detection mechanism on every node in the network allows achieving a perfect detection against DoS attacks but it is not a feasible solution in a constrained network.
%In
%\cite{HI09},
%an optimized placement of detection nodes in a network for distributed detection of DoS attacks is proposed.
%In addition to placing detection nodes at critical points in a network, this proposition minimizes the number of these required nodes and therefore reduces the cost and processing overheads.

%% Optimal Sensor Placement for Detection against Distributed Denial of Service Attacks
The best way to detect for sure a DoS attack in a WSN is simply to run a detection mechanism on each single sensor.
Of course, this solution is not feasible in a network with constraints.
Instead of fitting out each sensor with such mechanism, it is proposed in~\cite{INK09} to resort to heuristics in order to set a few nodes equipped with detection systems at critical spots in the network topology.
This optimized placement enables distributed detection of DoS attacks as well as reducing costs and processing overheads, since the number of required detectors is minimized.
But those few selected nodes are likely to run out of battery power much faster than normal nodes.


% ======== >>> §1 <<<
% Denial of service attack-resistant flooding authentication in wireless sensor networks
%\subsection{[JS10] -- Inchangé}

\textit{Son et al}. propose in
\cite{JS10}
a novel broadcast authentication mechanism to cope with DoS attacks in sensor networks.
This scheme uses an asymmetric distribution of keys between sensor nodes and the BS, and uses the Bloom filter as an authenticator, which efficiently compresses multiple authentication information.
In this model, the BS or sink shares symmetric keys with each sensor node, and proves its knowledge of the information through multiple MAC values in its flooding messages.
When the sink floods the network with control messages it constructs a Bloom filter as an authenticator for the message.
When a sensor node receives a flooded control message, it generates their Bloom filter with its keys and in the same way the sink verifies message authentication.

% Denial of service attack-resistant flooding authentication in wireless sensor networks
A novel broadcast authentication mechanism can also be deployed so as to cope with DoS attacks in sensor networks such as in~\cite{SLS10}.
This scheme uses an asymmetric distribution of keys between sensor nodes and the BS, and uses the Bloom filter as an authenticator, which efficiently compresses multiple authentication information.
In this model, the BS or sink shares symmetric keys with each sensor node, and proves its knowledge of the information through multiple MAC values in its flooding messages.
When the sink floods the network with control messages it constructs a Bloom filter as an authenticator for the message.
When a sensor node receives a flooded control message, it generates their Bloom filter with its keys and in the same way the sink verifies message authentication.




Several mechanisms have been proposed to detect it and to provide countermeasures\cite{SSS11,RM11}.
Even restricted to the data link, media access control and network layers of the network, there are many different existing \DoS attacks.
Consequently, even more solutions have been proposed to counter them.
Various studies have been realized in the purpose of countering one specific kind of attack, sometimes on specific protocols, thus leading to the proposals of hardened versions of AODV\cite{DLA02} or DSR\cite{CT04} for instance.
Some researchers prefer to ensure that nodes are not physically withdrawn from the network to get modified\cite{Ho10}, considering any returning node as compromised (but not detecting nodes modified on site), or to verify by means of mobile agents that the binary code run by the nodes has not been modified\cite{HR13}, although this solution makes use of cryptography mechanisms and can limit the evolution of the network.

Many other systems, often more convenient to deploy, consist in the implementation of trust based mechanisms\cite{MC10,F-GRL07} with agents applying set of rules\cite{RKKK13} on traffic to attribute a trust value to each of the nodes in the network.
Each enforced rule is expected to protect the network against one kind of attacks: intruders are detected on breaking the rules.
We are particularly interested in the solution of Lai and Chen\cite{LC08} who proposed to elect control nodes to monitor the traffic in clustered networks so as to detect and react to \dos attacks.
In such a scheme, clustered networks are partitioned into clusters \via algorithms such as \leach\cite{HHT02} or VSR\cite{TV08}.





To deal with DoS attacks in wireless sensor networks, many research studies have been conducted: there are many existing attacks able to compromise the good working of a \wsn.
Even more mechanisms have been proposed to detect it and to provide countermeasures~\cite{SSS11}.
Many consist in the implementation of trust based mechanisms~\cite{MC10,F-GRL07} with agents applying set of rules~\cite{RKKK13} on traffic to attribute a trust value to each of the nodes in the network.
Below are outlined some notable proposals.

% ======== >>> §2 <<<
%\subsection{[JH10] -- Ajouté}

% Distributed Detection of Node Capture Attacks in Wireless Networks
Some works examine the possibility to detect the compromising of nodes as soon as an opponent physically withdraw them from the network.
In the method that \textit{Ho} develops in
\cite{Ho10},
each node keeps watching on the presence of its neighbors.
The Sequential Probability Radio Test (SPRT) is used to determinate a dynamic time threshold.
When a node appears to be missing for a period longer that this threshold, it is considered to be dead or captured by an attacker.
If this node is later redeployed in the network, it will immediately be considered as compromised without having a chance to be harmful.
Nothing is done, however, if an attacker manages to compromise the node without extracting the sensor from its environment.

% Distributed Detection of Node Capture Attacks in Wireless Networks
Some works examine the possibility to detect the compromising of nodes as soon as an opponent physically withdraw them from the network.
In the method that is developed in~\cite{Ho10}, each node keeps watching on the presence of its neighbors.
The Sequential Probability Radio Test (SPRT) is used to determinate a dynamic time threshold.
When a node appears to be missing for a period longer that this threshold, it is considered to be dead or captured by an attacker.
If this node is later redeployed in the network, it will immediately be considered as compromised without having a chance to be harmful.
Nothing is done, however, if an attacker manages to compromise the node without extracting the sensor from its environment.

%% Using mobile agents to recover from node and database compromise in path-based DoS attacks in wireless sensor network
In~\cite{LB09} is described a method to detect and to recover from Path-based DoS (PDoS) attacks in wireless sensor networks.
The authors consider WSNs whose aim is to collect data and to store it into small databases.
PDoS attacks may prevent legitimate communication, lead the sensors to battery exhaustion and corrupt the gathered data.
Thus the use of Mobile Agents (MAs) is introduced: they use hash function values, node IDs and traffic table to analyze the traffic and identify compromised sensors.
In this way the MAs are able to detect PDoS attacks with ease and efficiency, and to reply to the attack by proceeding to a recovery process.
There are three distinct recovery processes available, depending on the percentage of compromised nodes in the network.
Note that the authors use the assumption that MAs can not be compromised.

%% Detecting denial of service attacks in sensor networks
Much of our work relies on the work of Lai and Chen who proposed in~\cite{LC08} a system detection based on static election of a set of nodes called ``guarding nodes'' which analyze traffic in a clustered network.
When detecting abnormal traffic from a given node, ``guarding nodes'' ---~we call them \cns~--- identify it as a compromised node and inform the \ch of it.
On reception of reports from several distinct \cns (to prevent false denunciation from a compromised node), the \CH virtually excludes the suspicious node from the cluster.
The authors show the benefit of their method by presenting numerical analysis of detection rate.
Although the method is efficient for detecting rogue nodes, the authors do not give details of the election mechanism to choose the \cns.
Also, there is no mention in their study of renewing the election in time, which causes the appointed \cns to endorse a heavier energy consumption on a long period.






% ======== >>> §5 <<<
%\subsection{[MH07] -- Modifié}

%% Adaptive security design with malicious node detection in cluster-based sensor networks
%% OLD VERSION
%An adaptive security design (SecCBSN) to secure cluster-based communication in sensor networks, is proposed by \textit{Hsieh et al.} in
%\cite{MH07}.
%It consists of three modules, which can detect malicious nodes by providing secure communication and authentication protocols between nodes.
%In each cluster, a CH schedules transmission and monitors periods for its sensor nodes.
%The primary security module uses the TESLA certificate (TCert) to enable existing nodes to authenticate new incoming nodes, triggering the establishment of secure links and broadcast authentication between neighboring nodes.
%In SecCBSN, the intrusion detection module prevents against compromised nodes.
%It uses alarm return protocols, trust value evaluation, and the propagation of black and white lists of nodes.

%% Adaptive security design with malicious node detection in cluster-based sensor networks
Sensors authentication and DoS detection in clustered networks may be assumed by a single architecture.
SecCBSN is described in~\cite{HHC07}.
It is an adaptive security design intended to Secure Cluster-Based Communication in Sensor Networks.
Each node is equipped with a system that includes three modules.
One is involved in the cluster head election, and responsible for remembering the decision which were made.
Another module provides ciphered communication and secure authentication protocols between sensors.
It uses the TESLA certificate to enable deployed sensors to authenticate new incoming nodes.
It allows the creation of secure channels as well as broadcast authentication between neighboring sensors.
The last security module is responsible for the detection of compromised nodes.
When a node is suspected to harm the network, alarm protocols are used to warn the base station.
The use of trust value evaluation then enables the setting and the propagation of black and white lists of sensors.







% ======== >>> §2 <<<
%\subsection{[LB09] -- Modifié}

%% Using mobile agents to recover from node and database compromise in path-based DoS attacks in wireless sensor network
\textit{Li} and \textit{Batten} expose in
\cite{LB09}
their method to detect and to recover from Path-based DoS (PDoS) attacks in wireless sensor networks.
They consider WSNs whose aim is to collect data and to store it into small databases.
PDoS attacks may prevent legitimate communication, lead the sensors to battery exhaustion and corrupt the gathered data.
So the authors introduce the use of Mobile Agents (MAs), which use hash function values, node IDs and traffic table to analyze the traffic and identify compromised sensors.
Thus the MAs are able to detect PDoS attacks with ease and efficiency, and to reply to the attack by proceeding to a recovery process.
There are three distinct recovery processes available, depending on the percentage of compromised nodes in the network.
Note that the authors use the assumption that MAs can not be compromised.
%% OLD VERSION
%A new detection and recovery method for compromised nodes in Path-based DoS (PDoS) attacks in wireless sensor network is presented in \cite{LB09} by \textit{Li} and \textit{Batten}. In this paper, the authors consider WSNs designed to collect and to store data in which path-based attacks can be carried on both the sensor nodes and the database containing the collected data. The contribution of this paper is the use of mobile agents (MAs) for the attack detection and the recovery process. The MAs detect PDoS attacks easily and efficiently. The solution is based on the use of hash function values, node IDs and traffic tables. By analyzing the traffic, the MA can detect the abnormal traffic and can designate the compromised nodes.


% ======== >>> §2 <<<
%\subsection{[LO07] -- Modifié}
%% OLD VERSION
%Hierarchical sensor networks where clusters are formed dynamically and periodically using the LEACH algorithm, allow to distribute the energy load among sensor nodes.
%The problem of adding security to this type of network is raised by \textit{Oliveira et al}
%\cite{LO07}
%who propose SecLEACH which is a revised version of LEACH.
%It applies random key pre-distribution and $\mu$TESLA.
%They can be used both to secure communications in a hierarchical network with dynamic cluster establishment.

%% SecLEACH -- On the security of clustered sensor networks
A sensor network may be recursively and periodically reclustered with an algorithm such as \leach, as in our proposal.
The resulting hierarchically clustered network often presents a good ability for distributing the energy consumption among the sensor nodes.
But security concerns (other than DoS) also apply to those networks.
In~\cite{OFVWBDL07}, the authors propose to add security mechanisms \via a revised version of \leach protocol.
\textit{SecLEACH}\ uses random key pre-distribution as well as $\mu$TESLA (authenticated broadcast) so as to protect communications.
But the authors do not mention any mechanism to fight DoS attacks.





Denial of service attacks embraces many different attacks, which can target all layers of the network~\cite{VS10}.
Jamming the radio frequencies as well as disturbing the routing protocols are just a few examples of ways to harm the network.
In reaction to these, a number of solutions have been proposed~\cite{SSS11}.

As stated in the introduction, we focus in this paper on inside attackers attempting to bend the MAC protocol parameters to their needs, be it to achieve better performances for themselves (\emph{greedy} attacks) or to generally harm the network (\emph{jamming} attacks or sleep deprivation).
To detect such attackers, many solutions rely on trust models~\cite{MC10}.
We base our own work on Lai's and Chen's approach~\cite{LC08}, which consists in assigning monitoring nodes in the network.
Those monitoring nodes, also called \cns in this article, apply a set of rules~\cite{RKKK13} to overheard traffic so as to detect misbehaving nodes.
On multiple rule breaks, they report the suspicious node to the \ch.
To prevent false positive detection, the \CH waits for several reports about a given node before considering it as compromised.
After that, it virtually excludes the misbehaving node from the cluster by broadcasting a warning to all sensors.
We have observed in previous studies that renewing periodically the selection process of those monitoring nodes (\cns) helps saving energy in the network~\cite{BMM13}, and we have tried ever since to find an optimal selection algorithm to obtain a good equilibrium between security, attack detection and energy preservation~\cite{MMB14}.
