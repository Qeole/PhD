% vim: set spelllang=fr foldmethod=marker:
\section{Sécurité dans les réseaux de capteurs}

\subsection{Plusieurs problématiques}

La sécurité des systèmes informatiques a pour particularité de constituer un domaine transversal à pratiquement tous les domaines de l'informatique, dans le sens où pour sécuriser la totalité d'un système, chaque couche, chaque technologie utilisée doit être sure.
Ainsi, la sécurité des réseaux constitue en soi tout un domaine d'étude, qui se décompose à son tour en plusieurs problématiques.
Les principales garanties que l'on cherche à apporter à un réseau sont les suivantes:
\begin{itemize}
    \item la confidentialité des communications, \cad qu'un attaquant ne doit pas être en mesure d'accéder au contenu utile des messages échangés;
    \item l'intégrité des messages, autrement dit l'assurance qu'ils n'ont pas été modifiés par un attaquant durant leur transfert;
    \item l'authentification des correspondants, propriété qui assure à un agent du réseau que son interlocuteur est bien celui qu'il prétend être;
    \item la non-répudiation, qui empêche un agent de nier l'envoi d'un message émis;
    \item la protection contre le rejeu vise à prévenir les attaques basées sur l'injection de paquets capturés au préalable par l'attaquant dans le but de générer du trafic supplémentaire;
    \item la sécurité physique des capteurs, nécessaire pour prévenir l'accès au contenu des communications ou au matériel cryptographique par exemple;
    \item la disponibilité, c'est à dire la résistance aux attaques de déni de service. Cette problématique est centrale pour les travaux présentés ici, et sera développée plus largement que les autres dans la \sref{ea:sec:dos};
\end{itemize}
À noter qu'il existe de nombreuses études, voire même des travaux de synthèse, qui se sont déjà concentrés sur la sécurité des réseaux de capteurs~\cite{DYK12,AD14}.

La sureté et la résilience des réseaux, traités en \ssref{ea:ssec:safety}, ne sont pas des problématiques de sécurité (car aucune attaque ne rentre en compte), mais il s'agit de problématiques assez proches de la disponibilité des réseaux.

\subsection{Confidentialité}

Les informations échangées sur un réseau sont dites confidentielles si un attaquant se retrouve dans l'impossibilité d'accéder au contenu utile de l'échange.
L'attaquant peut être passif (simplement à l'écoute du trafic) ou bien actif (injection de paquets pour déclencher l'envoi de paquets en réponse, détournement de trafic).
Pour de nombreux réseaux plus « classiques », l'utilisation d'un câble de transmission (Ethernet, fibre optique\dots) permet de poser une première barrière pour les écoutes passives.
Mais dans les réseaux de capteurs, les données sont échangées par le biais d'ondes électromagnétiques que tout attaquant à portée peut recevoir.
Il convient donc de trouver des solutions permettant de dissimuler le contenu des messages échangés, à défaut de prévenir l'accès aux messages eux-mêmes.

Les méthodes conventionnelles pour assurer la confidentialité des communications sont donc des solutions cryptographiques: des algorithmes de chiffrement symétrique (une même clé sert à chiffrer puis déchiffrer le message) ou bien asymétrique (une clé pour le chiffrement, une autre pour déchiffrer) sont utilisés dans tous types de communications pour empêcher l'accès à l'information à quiconque ne possède pas la clé adéquate.
\textsc{Aes} (\textit{Advanced Encryption Standard})~\cite{aes}, basé sur l'algorithme Rijndael, en est l'un des exemples les plus célèbres et les plus utilisés aujourd'hui.

Ces algorithmes de chiffrement, cependant, nécessitent en général de nombreux calculs, voire beaucoup de mémoire tampon, pour être appliqués, ce qui les rend parfois peu adaptés aux réseaux de capteurs.
Qui plus est, ils requièrent la plupart du temps la mise en place d'une architecture d'échange et de gestion du matériel cryptographique, notamment en ce qui concerne les échanges de clés: cette architecture peut considérablement alourdir la structure du réseau ainsi que les échanges en termes de volume de données de contrôle.

Plusieurs solutions adaptées aux capacités limitées des capteurs ont donc été proposées~\cite{OX09}.
Si certaines d'entre elles demeurent basées sur des algorithmes de chiffrement, qui se voient adaptés aux capacités des capteurs, d'autres mettent en place des mécanismes propres à profiter de l'architecture particulière de ces réseaux.




secure data aggregation without persistent cryptographic operations\cite{WDSX07}

Many other mechanisms related to authentication are resumed and compared in surveys such as~\cite{SOBMCN11}.



% ======== >>> §5 <<<
%\subsection{[MH07] -- Modifié}

%% Adaptive security design with malicious node detection in cluster-based sensor networks
%% OLD VERSION
%An adaptive security design (SecCBSN) to secure cluster-based communication in sensor networks, is proposed by \textit{Hsieh et al.} in
%\cite{MH07}.
%It consists of three modules, which can detect malicious nodes by providing secure communication and authentication protocols between nodes.
%In each cluster, a CH schedules transmission and monitors periods for its sensor nodes.
%The primary security module uses the TESLA certificate (TCert) to enable existing nodes to authenticate new incoming nodes, triggering the establishment of secure links and broadcast authentication between neighboring nodes.
%In SecCBSN, the intrusion detection module prevents against compromised nodes.
%It uses alarm return protocols, trust value evaluation, and the propagation of black and white lists of nodes.

%% Adaptive security design with malicious node detection in cluster-based sensor networks
Sensors authentication and DoS detection in clustered networks may be assumed by a single architecture.
SecCBSN is described in~\cite{HHC07}.
It is an adaptive security design intended to Secure Cluster-Based Communication in Sensor Networks.
Each node is equipped with a system that includes three modules.
One is involved in the cluster head election, and responsible for remembering the decision which were made.
Another module provides ciphered communication and secure authentication protocols between sensors.
It uses the TESLA certificate to enable deployed sensors to authenticate new incoming nodes.
It allows the creation of secure channels as well as broadcast authentication between neighboring sensors.
The last security module is responsible for the detection of compromised nodes.
When a node is suspected to harm the network, alarm protocols are used to warn the base station.
The use of trust value evaluation then enables the setting and the propagation of black and white lists of sensors.



% ======== >>> §2 <<<
%\subsection{[LB09] -- Modifié}

%% Using mobile agents to recover from node and database compromise in path-based DoS attacks in wireless sensor network
\textit{Li} and \textit{Batten} expose in
\cite{LB09}
their method to detect and to recover from Path-based DoS (PDoS) attacks in wireless sensor networks.
They consider WSNs whose aim is to collect data and to store it into small databases.
PDoS attacks may prevent legitimate communication, lead the sensors to battery exhaustion and corrupt the gathered data.
So the authors introduce the use of Mobile Agents (MAs), which use hash function values, node IDs and traffic table to analyze the traffic and identify compromised sensors.
Thus the MAs are able to detect PDoS attacks with ease and efficiency, and to reply to the attack by proceeding to a recovery process.
There are three distinct recovery processes available, depending on the percentage of compromised nodes in the network.
Note that the authors use the assumption that MAs can not be compromised.
%% OLD VERSION
%A new detection and recovery method for compromised nodes in Path-based DoS (PDoS) attacks in wireless sensor network is presented in \cite{LB09} by \textit{Li} and \textit{Batten}. In this paper, the authors consider WSNs designed to collect and to store data in which path-based attacks can be carried on both the sensor nodes and the database containing the collected data. The contribution of this paper is the use of mobile agents (MAs) for the attack detection and the recovery process. The MAs detect PDoS attacks easily and efficiently. The solution is based on the use of hash function values, node IDs and traffic tables. By analyzing the traffic, the MA can detect the abnormal traffic and can designate the compromised nodes.


% ======== >>> §2 <<<
%\subsection{[LO07] -- Modifié}
%% OLD VERSION
%Hierarchical sensor networks where clusters are formed dynamically and periodically using the LEACH algorithm, allow to distribute the energy load among sensor nodes.
%The problem of adding security to this type of network is raised by \textit{Oliveira et al}
%\cite{LO07}
%who propose SecLEACH which is a revised version of LEACH.
%It applies random key pre-distribution and $\mu$TESLA.
%They can be used both to secure communications in a hierarchical network with dynamic cluster establishment.

%% SecLEACH -- On the security of clustered sensor networks
A sensor network may be recursively and periodically reclustered with an algorithm such as \leach, as in our proposal.
The resulting hierarchically clustered network often presents a good ability for distributing the energy consumption among the sensor nodes.
But security concerns (other than DoS) also apply to those networks.
In~\cite{OFVWBDL07}, the authors propose to add security mechanisms \via a revised version of \leach protocol.
\textit{SecLEACH}\ uses random key pre-distribution as well as $\mu$TESLA (authenticated broadcast) so as to protect communications.
But the authors do not mention any mechanism to fight DoS attacks.




Some of these studies led to new protocols to provide confidentiality in WSNs, \ie to prevent unauthorized intruders to access to the meaningful data that nodes exchange in the network.
In~\cite{LPH08} for example, the authors introduce a method to ensure data confidentiality against parasitic adversaries.
The use a very simple key management scheme, roughly consisting in sharing a key between each node and the \bs.
They also introduce en route encryption mechanism to reinforce the confidentiality against compromised nodes: a special subset of nodes in the network re-encrypt the message on its way to the base station.
A periodic renewing of the keys of the node is deployed, to prevent attackers to use compromised keys at will.
The authors claim a low energy consumption in regard with classic public key encryption schemes.












Un usage intéressant de la cryptographie est le recours au chiffrement dit « homomorphique », qui permet d'effectuer certaines opérations sur des données chiffrées sans avoir besoin, justement, de les déchiffrer.
Proposés pour les réseaux de capteurs~\cite{BBTY14}, ces algorithmes permettent de procéder, dans une certaine mesure, à des opérations de compression et d'agrégation de données (par exemple) au cours du trajet d'un paquet, sans avoir à déchiffrer puis chiffrer à nouveau son contenu à chaque étape.

Mais à part les solutions de chiffrement, d'autres méthodes consistent plus simplement à compliquer l'accès aux paquets plutôt qu'à chiffrer le contenu.
Moins robustes, ces solutions sont aussi moins coûteuses en termes de calculs et, \infine, de consommation énergétique.

Certaines méthodes proposent de découper les paquets à envoyer en fragments, qui seront envoyés à travers le réseau par des routes distinctes.
Ces fragments seront combinés de telle sorte que seule la détention d'un certain nombre d'entre eux permette de reconstituer le message original.
Cette combination des fragments peut faire appel à des mécanismes très simples, par exemple un « ou exclusif » logique comme avec la méthode \textsc{Sdmp} (\textit{Securing Data based on Multi-Path routing})~\cite{BM10}, ou bien à un mécanisme un peu plus élaboré de partage de secret entre plusieurs entités, comme l'a proposé par exemple \textsc{Schamir}~\cite{Sha79}.
Ces méthodes ont pour intérêt de gêner l'attaquant pour la reconstruction des messages dont il ne dispose que de quelques fragments.
Elles sont inefficaces en revanche si l'attaquant contrôle toutes les routes du réseau, ou simplement s'il se trouve très proche géographiquement de l'émetteur ou du destinataire pour l'échange concerné.

Plusieurs de ces méthodes peuvent être associées, pour ajuster au mieux le compromis entre la sécurité et l'économie des ressources.
Nous avons ainsi proposé une solution pour assurer la confidentialité des échanges basée sur un système de priorités~\cite{MMB13}:
\begin{itemize}
    \item les paquets d'importance moindre peuvent être simplement découpés en fragment et envoyés par plusieurs routes distinctes selon la méthode \textsc{Sdmp};
    \item les paquets dont le contenu est plus important peuvent se voir rajouter une couche de sécurité, comme avec l'usage de la méthode de secret partagé;
    \item enfin, les messages d'importance critique, moins fréquents, utilisent des solutions plus lourdes, mais plus efficaces, de chiffrement cryptographique, qui seules sont à même d'apporter dans ce cas un degré convenable de sécurité.
\end{itemize}

Il est à noter qu'aucune des méthodes présentées ici, qu'elles reposent ou non sur l'usage de chiffrement, n'empêchent un attaquant de pouvoir récolter des métadonnées sur les échanges effectués, et d'en déduire par exemple qui communique avec qui, ou quel est le volume des données échangées.
Ces métadonnées n'ont que rarement une valeur importante dans les réseaux de capteurs, et peuvent être en partie masquées, au besoin, par l'envoi de faux messages supplémentaires pour « brouiller les pistes », mais cela s'avèrerait très coûteux pour le réseau.
Néanmoins certaines études considèrent que toute information peut avoir de la valeur, et se proposent par exemple de dissimuler l'emplacement géographique des différents nœuds d'un réseau de capteurs, que ce soit à travers l'usage d'algorithmes de clustering~\cite{GK13} ou bien de dissémination de l'information sur le modèle des épidémies~\cite{KDA14}.

\subsection{Intégrité, authentification, non-répudiation, protection contre le rejeu}

Assurer l'intégrité d'un message revient à s'assurer que le contenu utile du message n'a pas été altéré ou amputé entre sa création par l'émetteur et la réception par le destinataire final.
Des méthodes cryptographiques peuvent être employées pour créer un condensat ou (s'il est utilisé avec une clé symétrique) un code d'authentification (\textsc{Mac} en anglais, pour \textit{Message Authentication Code}) des messages, permettant cette vérification.

L'authentification sert à prouver la validité de l'expéditeur des messages.
Il n'est pas suffisant, pour assurer la sécurité d'un réseau, de s'assurer que le contenu des messages est chiffré: il faut également s'assurer que l'entité avec laquelle on communique est bien qui elle prétend être.
Cette précaution permet notamment de se prémunir de l'attaque de \textit{l'homme du milieu}.
L'authentification regroupe en réalité à la fois l'authentification de deux parties l'une envers l'autre, et l'authentification d'un agent aux yeux de tous, dans le cas d'une diffusion générale (\textit{broadcast}) des messages.
La signature des messages est le plus souvent réalisée à l'aide de mécanismes cryptographiques asymétriques (condensat signé à l'aide d'une clé privée).

La non-répudiation consiste à assurer qu'un agent du réseau ne peut plus nier avoir émis un paquet une fois celui-ci reçu par son destinataire.
Concrètement, si un agent est le seul à posséder une clé privée utilisée pour la signature d'un paquet, il ne fait nul doute, une fois ce paquet reçu, que cet agent en question est bien à l'origine de la création du paquet, puisque nul autre agent du réseau n'est en mesure de le signer à sa place.
L'usage de ces dispositifs de signature en font un aspect étroitement liée à l'authentification des messages.
Indispensable dans les transactions en ligne, la non-répudiation n'est pas toujours essentielle dans les réseaux de capteurs, mais peut s'avérer utile pour identifier de manière certaine l'auteur de comportements malveillants dans le réseau.

Les attaques par rejeu consistent pour un attaquant à capturer des paquets, même chiffrés, en transit sur le réseau, puis à les injecter à nouveau dans le réseau.
Le but visé peut être de générer davantage de trafic, dans le but d'augmenter le volume de données capturées.
Il s'agit par exemple de la méthode utilisée pour accélérer considérablement l'obtention de la clé d'accès à un réseau sans fil sécurisé à l'aide de la suite WEP (\textit{Wired Equivalent Privacy})~\cite{BGW01}.
Un autre but recherché peut être de générer du trafic superflu dans le but de consommer inutilement les ressources du réseau, ce qui en fait potentiellement une attaque de déni de service.
Pour se prévenir de ce genre d'attaques, des moyens cryptographiques peuvent ici encore être mis en œuvre pour associer un identifiant unique à chaque paquet, de sorte qu'un paquet rejoué soit rejeté par les agents du réseau.

Ces différentes propriétés sont distinctes, mais reposent presque toujours sur l'usage de solution cryptographiques pour être préservées dans un réseau.
Elles sont souvent liées, et liées à la confidentialité des données, car il est fréquent qu'un dispositif mis en place pour assurer l'une de ces propriétés permette également d'en assurer d'autre.
Par exemple, l'usage d'un condensat signé permet d'assurer à la fois l'intégrité d'un message, la validité de son auteur, et d'empêcher ce dernier de répudier le message dans le futur.

Ainsi \textsc{Spins} (\textit{Security Protocols for Sensor Networks}), élaboré en 2002~\cite{PSWCT02}, est l'une des premières solutions proposées dans la littérature pour sécuriser les échanges dans les réseaux de capteurs, et se propose de répondre à l'ensemble de ces problématiques.
Il s'agit d'un système reposant sur deux blocs: \textsc{Snep} (\textit{Secure Network Encryption Protocol}) fournit la confidentialité, l'authentification entre deux parties, l'intégrité et la protection contre le rejeu, le tout en ajoutant peu de données de contrôle aux paquets.
Des codes d'authentification des messages associés à un compteur partagé entre les interlocuteurs, et utilisé pour le bloc de chiffrement (le compteur est incrémenté après chaque bloc), sont employés à cette fin.
Le second bloc est appelé $\mu$\textsc{Tesla} (version « micro » du \textit{Timed, Efficient, Streaming, Loss-tolerant Authentication Protocol}) fournit des diffusions générales (\textit{broadcast}) authentifiées à l'aide de chaînes de clés: ces chaînes à sens uniques fournissent des clés associées à chaque intervalle de temps pour la signature des messages.




Authentication has been deeply investigating as well~\cite{GWZC13}.
Many lightweight proposals for key management in WSNs have been suggested~\cite{GWZCK13,BSK13}. % #NotTheSame







\subsection{Sécurité physique}

