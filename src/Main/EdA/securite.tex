% vim: set spelllang=fr foldmethod=marker:
\section{Sécurité dans les réseaux de capteurs}\label{ea:sec:secu}

\subsection{Plusieurs problématiques}

La sécurité des systèmes informatiques a pour particularité de constituer un domaine transversal à pratiquement tous les autres domaines de l'informatique, dans le sens où pour sécuriser la totalité d'un système, chaque couche, chaque technologie utilisée doit être sure.
Ainsi, la sécurité des réseaux constitue en soi tout un domaine d'étude, qui se décompose à son tour en plusieurs problématiques.
Les principales garanties que l'on cherche à apporter à un réseau sont les suivantes:
\begin{itemize}
    \item la confidentialité des communications, \cad qu'un attaquant ne doit pas être en mesure d'accéder au contenu utile des messages échangés;
    \item l'intégrité des messages, autrement dit l'assurance qu'ils n'ont pas été modifiés par un attaquant durant leur transfert;
    \item l'authentification des correspondants, propriété qui assure à un agent du réseau que son interlocuteur est bien celui qu'il prétend être;
    \item la non-répudiation, qui empêche un agent de nier l'envoi d'un message émis;
    \item la protection contre le rejeu vise à prévenir les attaques basées sur l'injection de paquets capturés au préalable par l'attaquant dans le but de générer du trafic supplémentaire;
    \item la sécurité physique des capteurs, nécessaire pour prévenir l'accès au contenu des communications ou au matériel cryptographique par exemple;
    \item la disponibilité, c'est à dire la résistance aux attaques de déni de service. Cette problématique est centrale pour les travaux présentés ici, et sera développée plus largement que les autres dans la \sref{ea:sec:dos};
\end{itemize}
À noter qu'il existe de nombreuses études, voire même des travaux de synthèse, qui se sont déjà concentrés sur la sécurité des réseaux de capteurs~\cite{DYK12,AD14}.

\subsection{Confidentialité}

Les informations échangées sur un réseau sont dites confidentielles si un attaquant se retrouve dans l'impossibilité d'accéder au contenu utile de l'échange.
L'attaquant peut être passif (simplement à l'écoute du trafic) ou bien actif (injection de paquets pour déclencher l'envoi de paquets en réponse, détournement de trafic).
Pour de nombreux réseaux plus « classiques », l'utilisation d'un câble de transmission (Ethernet, fibre optique\dots) permet de poser une première barrière pour les écoutes passives.
Mais dans les réseaux de capteurs, les données sont échangées par le biais d'ondes électromagnétiques que tout attaquant à portée peut recevoir.
Il convient donc de trouver des solutions permettant de dissimuler le contenu des messages échangés, à défaut de prévenir l'accès aux messages eux-mêmes.

Les méthodes conventionnelles pour assurer la confidentialité des communications sont donc des solutions cryptographiques: des algorithmes de chiffrement symétrique (une même clé sert à chiffrer puis déchiffrer le message) ou bien asymétrique (une clé pour le chiffrement, une seconde pour déchiffrer) sont utilisés dans tous types de communications pour empêcher l'accès à l'information à quiconque ne possède pas la clé adéquate.
\textsc{Aes} (\textit{Advanced Encryption Standard})~\cite{aes}, basé sur l'algorithme Rijndael, en est l'un des exemples les plus célèbres et les plus utilisés aujourd'hui.
Certaines études se sont penchées sur les vitesses d'exécution relatives de différents algorithmes utilisés pour le chiffrement et l'authentification des données~\cite{SOBMCN11}.

Ces algorithmes de chiffrement nécessitent en général de nombreux calculs, voire beaucoup de mémoire tampon, pour être mis en œuvre, surtout lorsque l'appareil ne dispose pas de matériel dédié~\cite{PLP06}, ce qui les rend parfois peu adaptés aux réseaux de capteurs.
Qui plus est, ils requièrent la plupart du temps la mise en place d'une architecture d'échange et de gestion du matériel cryptographique, notamment en ce qui concerne les échanges de clés: cette architecture peut considérablement alourdir la structure du réseau ainsi que les échanges en termes de volume de données de contrôle.
Plusieurs solutions adaptées aux capacités limitées des capteurs ont donc été proposées~\cite{OX09}.
Certaines d'entre elles demeurent basées sur des suites de protocoles dont les algorithmes de chiffrement sont choisis au mieux pour répondre aux capacités des capteurs~\cite{SOBMCN11}; ces solutions couvrent le plus souvent d'autres problématiques que la seule confidentialité des données, et certaines seront présentées dans la sous-section suivante.
D'autres mettent en place des mécanismes propres à profiter de l'architecture particulière de ces réseaux, en voici certains exemples.

Dans le cas d'un attaquant « parasite » qui chercherait à exploiter les mesures relevées par le réseau de capteurs déployé par d'autres, des solutions comme GossiCrypt~\cite{LPH08}, basée sur un algorithme simple d'échanges de clés entre les nœuds et la station de base, ainsi que sur un renouvellement probabiliste du chiffrement: à chaque étape du message dans le réseau, le nœud relai charger de renvoyer le message choisit aléatoirement de chiffrer à son tour, ou non, le contenu du paquet.
Les clés de chiffrement des nœuds sont renouvelées régulièrement, mais pas toutes à la fois, de telle sorte qu'un nœud compromis par un attaquant ne devrait pouvoir déchiffrer le contenu que d'un nombre restreints de paquets.

Un usage intéressant de la cryptographie est le recours au chiffrement dit « homomorphique », qui permet d'effectuer certaines opérations sur des données chiffrées sans avoir besoin, justement, de les déchiffrer.
Proposés pour les réseaux de capteurs~\cite{BBTY14}, ces algorithmes permettent de procéder, dans une certaine mesure, à des opérations de compression et d'agrégation de données (par exemple) au cours du trajet d'un paquet, sans avoir à déchiffrer puis chiffrer à nouveau son contenu à chaque étape.

À part les solutions de chiffrement, d'autres méthodes consistent plus simplement à compliquer l'accès aux paquets plutôt qu'à chiffrer le contenu.
Moins robustes, ces solutions sont aussi moins coûteuses en termes de calculs et, \infine, de consommation énergétique.
Certaines méthodes proposent ainsi de découper les paquets à envoyer en fragments, qui seront envoyés à travers le réseau par des routes distinctes.
Ces fragments seront combinés de telle sorte que seule la détention d'un certain nombre d'entre eux permette de reconstituer le message original.
Cette combination des fragments peut faire appel à des mécanismes très simples, par exemple un « ou exclusif » logique comme avec la méthode \textsc{Sdmp} (\textit{Securing Data based on Multi-Path routing})~\cite{BM10}, ou bien à un mécanisme un peu plus élaboré de partage de secret entre plusieurs entités, comme l'a proposé par exemple \textsc{Schamir}~\cite{Sha79}.
Ces méthodes ont pour intérêt de gêner l'attaquant pour la reconstruction des messages dont il ne dispose que de quelques fragments.
Elles sont inefficaces en revanche si l'attaquant contrôle toutes les routes du réseau, ou simplement s'il se trouve très proche géographiquement de l'émetteur ou du destinataire pour l'échange concerné.
Plusieurs de ces méthodes peuvent être associées, pour ajuster au mieux le compromis entre la sécurité et l'économie des ressources.
Nous avons ainsi proposé une solution pour assurer la confidentialité des échanges basée sur un système de priorités~\cite{MMB13}:
\begin{itemize}
    \item les paquets d'importance moindre peuvent être simplement découpés en fragment et envoyés par plusieurs routes distinctes selon la méthode \textsc{Sdmp};
    \item les paquets dont le contenu est plus important peuvent se voir rajouter une couche de sécurité, comme avec l'usage de la méthode de secret partagé;
    \item enfin, les messages d'importance critique, moins fréquents, utilisent des solutions plus lourdes, mais plus efficaces, de chiffrement cryptographique, qui seules sont à même d'apporter dans ce cas un degré convenable de sécurité.
\end{itemize}

Il est à noter qu'aucune des méthodes présentées ici, qu'elles reposent ou non sur l'usage de chiffrement, n'empêchent un attaquant de pouvoir récolter des métadonnées sur les échanges effectués, et d'en déduire par exemple qui communique avec qui, ou quel est le volume des données échangées.
Ces métadonnées n'ont que rarement une valeur importante dans les réseaux de capteurs, et peuvent être en partie masquées, au besoin, par l'envoi de faux messages supplémentaires pour « brouiller les pistes », mais cela s'avèrerait très coûteux pour le réseau.

\subsection{Intégrité, authentification, non-répudiation, protection contre le rejeu}

Assurer l'intégrité d'un message revient à s'assurer que le contenu utile du message n'a pas été altéré ou amputé entre sa création par l'émetteur et la réception par le destinataire final.
Des méthodes cryptographiques peuvent être employées pour créer un condensat ou (s'il est utilisé avec une clé symétrique) un code d'authentification (\textsc{Mac} en anglais, pour \textit{Message Authentication Code}) des messages, permettant cette vérification.

L'authentification sert à prouver la validité de l'expéditeur des messages.
Il n'est pas suffisant, pour assurer la sécurité d'un réseau, de s'assurer que le contenu des messages est chiffré: il faut également s'assurer que l'entité avec laquelle on communique est bien qui elle prétend être.
Cette précaution permet notamment de se prémunir de l'attaque de \textit{l'homme du milieu}.
L'authentification regroupe en réalité à la fois l'authentification de deux parties l'une envers l'autre, et l'authentification d'un agent aux yeux de tous, dans le cas d'une diffusion générale (\textit{broadcast}) des messages.
La signature des messages est le plus souvent réalisée à l'aide de mécanismes cryptographiques asymétriques (condensat signé à l'aide d'une clé privée).

La non-répudiation consiste à assurer qu'un agent du réseau ne peut plus nier avoir émis un paquet une fois celui-ci reçu par son destinataire.
Concrètement, si un agent est le seul à posséder une clé privée utilisée pour la signature d'un paquet, il ne fait nul doute, une fois ce paquet reçu, que cet agent en question est bien à l'origine de la création du paquet, puisque nul autre agent du réseau n'est en mesure de le signer à sa place.
L'usage de ces dispositifs de signature en font un aspect étroitement liée à l'authentification des messages.
Indispensable dans les transactions en ligne, la non-répudiation n'est pas toujours essentielle dans les réseaux de capteurs, mais peut s'avérer utile pour identifier de manière certaine l'auteur de comportements malveillants dans le réseau.

Les attaques par rejeu consistent pour un attaquant à capturer des paquets, même chiffrés, en transit sur le réseau, puis à les injecter à nouveau dans le réseau.
Le but visé peut être de générer davantage de trafic, dans le but d'augmenter le volume de données capturées.
Il s'agit par exemple de la méthode utilisée pour accélérer considérablement l'obtention de la clé d'accès à un réseau sans fil sécurisé à l'aide de la suite WEP (\textit{Wired Equivalent Privacy})~\cite{BGW01}.
Un autre but recherché peut être de générer du trafic superflu dans le but de consommer inutilement les ressources du réseau, ce qui en fait potentiellement une attaque de déni de service.
Pour se prévenir de ce genre d'attaques, des moyens cryptographiques peuvent ici encore être mis en œuvre pour associer un identifiant unique à chaque paquet, de sorte qu'un paquet rejoué soit rejeté par les agents du réseau.

Ces différentes propriétés sont distinctes, mais elles reposent presque toujours sur l'usage de solution cryptographiques pour être préservées dans un réseau.
Elles sont souvent liées, et liées à la confidentialité des données, car il est fréquent qu'un dispositif mis en place pour assurer l'une de ces propriétés permette également d'en assurer d'autre.
Par exemple, l'usage d'un condensat signé permet d'assurer à la fois l'intégrité d'un message, la validité de son auteur, et d'empêcher ce dernier de répudier le message dans le futur.

Ainsi \textsc{Spins} (\textit{Security Protocols for Sensor Networks}), élaboré en 2002~\cite{PSWCT02}, est l'une des premières solutions proposées dans la littérature pour sécuriser les échanges dans les réseaux de capteurs, et se propose de répondre à l'ensemble de ces problématiques.
Il s'agit d'un système reposant sur deux blocs: \textsc{Snep} (\textit{Secure Network Encryption Protocol}) fournit la confidentialité, l'authentification entre deux parties, l'intégrité et la protection contre le rejeu, le tout en ajoutant peu de données de contrôle aux paquets.
Des codes d'authentification des messages associés à un compteur partagé entre les interlocuteurs, et utilisé pour le bloc de chiffrement (le compteur est incrémenté après chaque bloc), sont employés à cette fin.
Le second bloc est appelé $\mu$\textsc{Tesla} (version « micro » du \textit{Timed, Efficient, Streaming, Loss-tolerant Authentication Protocol}) fournit des diffusions générales (\textit{broadcast}) authentifiées à l'aide de chaînes de clés: ces chaînes à sens uniques fournissent des clés associées à chaque intervalle de temps pour la signature des messages.

Une étude de 2007 s'intéresse à plusieurs architectures de sécurité compatibles avec la suite de protocoles \textsc{IEEE} 802.15.4, utilisée dans certains réseaux sans fil et parfois avec les capteurs~\cite{BN07}.
Les modèles considérés sont \textsc{Spins}, présenté ci-dessus, \textsc{Leap}, \textsc{TinySec}, ZigBee (notamment dans son mode « commercial ») et Security Manager.
Sans rentrer dans les détails de fonctionnement de ces solutions, la \tabref{ea:tab:proto} compare brièvement les fonctionnalités apportées par ces différentes modèles.
\begin{table}[!ht]
    \caption{Brève comparaison d'architectures de sécurité pour réseaux de capteurs}\label{ea:tab:proto}
    \medskip
    \centering
    \newcolumntype{X}{>{\centering\arraybackslash}m}
    \begin{footnotesize}
        \begin{tabular}{X{.12\textwidth}|X{.10\textwidth} X{.13\textwidth} X{.11\textwidth} X{.11\textwidth} X{.21\textwidth}}
            \toprule
            Architecture & Chiffrement & Code d'authentification des messages & Protection contre le rejeu & Données de contrôle & Échange de clés\tabularnewline
            \toprule
            \textsc{Spins} & oui & oui & 8 octets & oui & chaînes de clés symétriques\tabularnewline
            \midrule
            \textsc{Leap} & oui & non & variable & oui & clés pré-déployées, renouvelables\tabularnewline
            \midrule
            \textsc{TinySec} & oui & non & 4 octets & oui & au choix\tabularnewline
            \midrule
            ZigBee (mode commercial) & oui & oui & 4, 8 ou 16 octets & oui & « centre de confiance »\tabularnewline
            \midrule
            Security Manager & oui & non & variable & oui & \textsc{Ec-Mvq} (courbes elliptiques); confiance initiale\tabularnewline
            \bottomrule
        \end{tabular}
    \end{footnotesize}
\end{table}

Une solution intéressante pour économiser des ressources consiste à n'introduire l'usage de solutions cryptographiques que lorsqu'elles deviennent nécessaires, et à s'en dispenser lorsque le réseau fonctionne correctement.
Ce système a été proposé en association avec l'usage d'un « arbre d'agrégation sécurisée », structure du réseau utilisée pour le routage des paquets~\cite{WDSX07}.
Lorsqu'un comportement malveillant est détecté dans le réseau, cet arbre permet de mettre en place rapidement et avec peu de données de contrôle un mécanisme d'authentification et de chiffrement des messages.
Le danger de cette proposition repose sur la capacité du réseau à détecter les attaques: si aucun comportement suspect n'est constaté, aucune mesure de sécurité n'est mise en place.
En cas d'écoute passive de la part de l'attaquant notamment, aucune mesure n'est appliquée pour protéger la confidentialité des données échangées.

Plusieurs propositions concernent la mise en place de solutions classiques, mais « allégées », comme avec la mise en place d'une autorité de certification spécifique permettant l'authentification réciproque des nœuds deux à deux~\cite{GWZCK13}, ou bien basé sur des clés pré-distribuées~\cite{BSK13}.
Le protocole MSPQS (\textit{Modified Secured Query Processing Scheme}~\cite{GD14} a été proposé pour sécurisé le transfert des données à travers un découpage des échanges sous formes de requêtes et de réponses authentifiées qui contiennent des marqueurs temporels, et ne peuvent être réinjectés dans le réseau, ce qui le rend efficace pour protéger les nœuds des attaques par rejeu.

Les problèmes d'authentification, et dans une moindre mesure les autres problématiques soulevées dans cette sous-section, ont également fait l'objet de plusieurs études~\cite{GWZC13}.

\subsection{Sécurité physique}

L'accès au réseau pour un attaquant peut passer par la compromission de l'un des nœud du réseau.
Dans le pire des cas, tout le système peut être modifié et détourné de son but d'origine, pour mener d'autres attaques par la suite, tandis que la mémoire ---~et notamment le matériel cryptographique embarqué~--- peut être entièrement analysée.

Les systèmes d'exploitation utilisés par les capteurs sont généralement conçus pour être légers et s'adapter aux contraintes des capteurs.
Mais ils ne rajoutent pas de problématiques particulières en termes de sécurité système par rapport à d'autres machines, et le peu de fonctionnalités dont ils disposent joue plutôt en leur faveur pour ce qui est de la sécurité logicielle.
Leur faiblesse réside donc dans leur déploiement en milieu ouvert et potentiellement hostile, ce qui peut permettre aux attaquants d'y accéder physiquement et de manière prolongée.

L'une des solutions proposées consiste donc à considérer un nœud comme ayant épuisé sa batterie s'il ne répond plus pendant une période prolongée; si en revanche le capteur redevient actif par la suite, on suppose que son absence était due à son extraction du réseau par un attaquant, et le capteur est alors considéré comme compromis~\cite{Ho10}.

La compromission d'un nœud peut être extrêmement nuisible pour le réseau de capteurs.
Non content d'accéder aux données reçues par ce nœud, ou aux clés de chiffrement et de signature qu'il peut contenir, l'attaquant peut en plus chercher à mener des attaques de déni de service visant à perturber le bon fonctionnement du réseau.
