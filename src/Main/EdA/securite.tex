% vim: set spelllang=fr foldmethod=marker:
\section{Sécurité dans les réseaux de capteurs}

\subsection{Plusieurs problématiques}

\subsection{Confidentialité}

\subsection{Intégrité}

\subsection{Authentification}

\subsection{Non répudiation}






For sensitive operations involving the deployment of a \wsn, all security aspects of the network must be reviewed.
WSN-addressed protocols to provide data privacy\cite{OX09} and authentication\cite{SOBMCN11} have been subject to deep research investigation, and led for example to the proposal of mechanisms for secure data aggregation without persistent cryptographic operations\cite{WDSX07} or for authenticated broadcast such as $\mu$TESLA\cite{PSWCT02}.
But cryptography is of no use if the network is down: in this paper we focused on resistance against \dos (DoS) attacks.

In a first attempt to bring load balancing to this solution, we propose in other papers\cite{GMT12,BMM13} to reiterate the election periodically.
Simulations show a better load repartition among the cluster, but at this time our focus was not on designing an energy efficient election process for the \cns.



% ======== >>> §1 <<<
%\subsection{[AP01] -- Modifié}

%% SPINS: Security Protocols for Sensor Networks
Back in 2001, most works focused on making WSNs feasible and useful.
But some people already involved themselves into security.
For instance, \textit{Perrig et al}. proposed SPINS (Security Protocols for Sensor Networks) in
\cite{PSWCT02}
to provide networks with two symmetric key-based security building blocks.
The first block, called SNEP (Secure Network Encryption Protocol), provides data confidentiality, two-party data authentication and data freshness.
The second block, called $\mu$TESLA (``micro'' version of the Timed, Efficient, Streaming, Loss-tolerant Authentication Protocol) assumes authenticated broadcast using one-way key chains constructed with secure hash functions.
No mechanism was put forward to detect DoS attacks.

%% OLD VERSION
%\textit{Perrig et al}.
%\cite{AP01}
%proposed SPINS (Security Protocols for Sensor Networks) which include two efficient symmetric key based security building blocks: SNEP and $\mu$TESLA.
%SNEP provides data confidentiality, two-party data authentication, and data freshness, with low overhead.
%It uses MAC and a shared counter between the sender and the receiver for the cipher block in counter mode which is incremented after each block to achieve two-party authentication and data integrity.
%$\mu$TESLA provides authenticated broadcast using one-way key chains constructed with secure hash functions.
%It divides a time into time intervals and the sender associates each key of the one-way key chain with one time interval to do cryptographic operations.

%% SPINS: Security Protocols for Sensor Networks
Back in 2001, most works focused on making WSNs feasible and useful.
But some people already involved themselves into security.
For instance, SPINS (Security Protocols for Sensor Networks) was proposed in~\cite{PSWCT02} to provide networks with two symmetric key-based security building blocks.
The first block, called SNEP (Secure Network Encryption Protocol), provides data confidentiality, two-party data authentication and data freshness.
The second block, called $\mu$TESLA (``micro'' version of the Timed, Efficient, Streaming, Loss-tolerant Authentication Protocol) assumes authenticated broadcast using one-way key chains constructed with secure hash functions.
No mechanism was put forward to detect DoS attacks.


% ======== >>> §5 <<<
%\subsection{[MH07] -- Modifié}

%% Adaptive security design with malicious node detection in cluster-based sensor networks
Sensors authentication and DoS detection in clustered networks may be assumed by a single architecture.
In
\cite{HHC07},
\textit{Hsieh et al}. present SecCBSN, an adaptive security design intended to Secure Cluster-Based Communication in Sensor Networks.
Each node is equipped with a system that includes three modules.
One is involved in the cluster head election, and responsible for remembering the decision which were made.
Another module provides ciphered communication and secure authentication protocols between sensors.
It uses the TESLA certificate to enable deployed sensors to authenticate new incoming nodes.
It allows the creation of secure channels as well as broadcast authentication between neighboring sensors.
The last security module is responsible for the detection of compromised nodes.
When a node is suspected to harm the network, alarm protocols are used to warn the base station.
The use of trust value evaluation then enables the setting and the propagation of black and white lists of sensors.

%% OLD VERSION
%An adaptive security design (SecCBSN) to secure cluster-based communication in sensor networks, is proposed by \textit{Hsieh et al.} in
%\cite{MH07}.
%It consists of three modules, which can detect malicious nodes by providing secure communication and authentication protocols between nodes.
%In each cluster, a CH schedules transmission and monitors periods for its sensor nodes.
%The primary security module uses the TESLA certificate (TCert) to enable existing nodes to authenticate new incoming nodes, triggering the establishment of secure links and broadcast authentication between neighboring nodes.
%In SecCBSN, the intrusion detection module prevents against compromised nodes.
%It uses alarm return protocols, trust value evaluation, and the propagation of black and white lists of nodes.

%% Adaptive security design with malicious node detection in cluster-based sensor networks
Sensors authentication and DoS detection in clustered networks may be assumed by a single architecture.
SecCBSN is described in~\cite{HHC07}.
It is an adaptive security design intended to Secure Cluster-Based Communication in Sensor Networks.
Each node is equipped with a system that includes three modules.
One is involved in the cluster head election, and responsible for remembering the decision which were made.
Another module provides ciphered communication and secure authentication protocols between sensors.
It uses the TESLA certificate to enable deployed sensors to authenticate new incoming nodes.
It allows the creation of secure channels as well as broadcast authentication between neighboring sensors.
The last security module is responsible for the detection of compromised nodes.
When a node is suspected to harm the network, alarm protocols are used to warn the base station.
The use of trust value evaluation then enables the setting and the propagation of black and white lists of sensors.



% ======== >>> §2 <<<
%\subsection{[LB09] -- Modifié}

%% Using mobile agents to recover from node and database compromise in path-based DoS attacks in wireless sensor network
\textit{Li} and \textit{Batten} expose in
\cite{LB09}
their method to detect and to recover from Path-based DoS (PDoS) attacks in wireless sensor networks.
They consider WSNs whose aim is to collect data and to store it into small databases.
PDoS attacks may prevent legitimate communication, lead the sensors to battery exhaustion and corrupt the gathered data.
So the authors introduce the use of Mobile Agents (MAs), which use hash function values, node IDs and traffic table to analyze the traffic and identify compromised sensors.
Thus the MAs are able to detect PDoS attacks with ease and efficiency, and to reply to the attack by proceeding to a recovery process.
There are three distinct recovery processes available, depending on the percentage of compromised nodes in the network.
Note that the authors use the assumption that MAs can not be compromised.

%% OLD VERSION
%A new detection and recovery method for compromised nodes in Path-based DoS (PDoS) attacks in wireless sensor network is presented in \cite{LB09} by \textit{Li} and \textit{Batten}. In this paper, the authors consider WSNs designed to collect and to store data in which path-based attacks can be carried on both the sensor nodes and the database containing the collected data. The contribution of this paper is the use of mobile agents (MAs) for the attack detection and the recovery process. The MAs detect PDoS attacks easily and efficiently. The solution is based on the use of hash function values, node IDs and traffic tables. By analyzing the traffic, the MA can detect the abnormal traffic and can designate the compromised nodes.


% ======== >>> §2 <<<
%\subsection{[LO07] -- Modifié}

%% SecLEACH -- On the security of clustered sensor networks
A sensor network may be recursively and periodically reclustered with an algorithm such as LEACH, as in our proposal.
The resulting hierarchically clustered network often presents a good ability for distributing the energy consumption among the sensor nodes.
But security concerns (other than DoS) also apply to those networks.
In
\cite{OFVWBDL07},
\textit{Oliveira et al}. propose to add security mechanisms \textit{via} a revised version of LEACH protocol.
SecLEACH uses random key pre-distribution as well as $\mu$TESLA (authenticated broadcast) so as to protect communications.
But the authors do not mention any mechanism to fight DoS attacks.

%% OLD VERSION
%Hierarchical sensor networks where clusters are formed dynamically and periodically using the LEACH algorithm, allow to distribute the energy load among sensor nodes.
%The problem of adding security to this type of network is raised by \textit{Oliveira et al}
%\cite{LO07}
%who propose SecLEACH which is a revised version of LEACH.
%It applies random key pre-distribution and $\mu$TESLA.
%They can be used both to secure communications in a hierarchical network with dynamic cluster establishment.

%% SecLEACH -- On the security of clustered sensor networks
A sensor network may be recursively and periodically reclustered with an algorithm such as \leach, as in our proposal.
The resulting hierarchically clustered network often presents a good ability for distributing the energy consumption among the sensor nodes.
But security concerns (other than DoS) also apply to those networks.
In~\cite{OFVWBDL07}, the authors propose to add security mechanisms \via a revised version of \leach protocol.
\textit{SecLEACH}\ uses random key pre-distribution as well as $\mu$TESLA (authenticated broadcast) so as to protect communications.
But the authors do not mention any mechanism to fight DoS attacks.



Deploying \wsns for sensitive operations requires that all aspects of network security are reviewed.
Hence numerous research investigations have been undertaken on ways to improve security in WSNs.

Some of these studies led to new protocols to provide confidentiality in WSNs, \ie to prevent unauthorized intruders to access to the meaningful data that nodes exchange in the network.
In~\cite{LPH08} for example, the authors introduce a method to ensure data confidentiality against parasitic adversaries.
The use a very simple key management scheme, roughly consisting in sharing a key between each node and the \bs.
They also introduce en route encryption mechanism to reinforce the confidentiality against compromised nodes: a special subset of nodes in the network re-encrypt the message on its way to the base station.
A periodic renewing of the keys of the node is deployed, to prevent attackers to use compromised keys at will.
The authors claim a low energy consumption in regard with classic public key encryption schemes.
But encryption itself is not always necessary: another example concerning confidentiality is~\cite{MMB13}, where we proposed several ways to split messages and to send them through several distinct paths in the network to avoid using systematical strong encryption.
Depending on the importance of the data contained in each packet, we propose to send it using one of the three: the SDMP method\cite{BM10}, a secret sharing scheme such as Shamir's\cite{Sha79}, or standard encryption.

Another issue consists in ensuring that every node actually is who it pretends to be (thus preventing address spoofing).
This is called authentication.
It often comes along with data integrity.
For instance the proposals in~\cite{PSWCT02} are based on two symmetric key-based security building blocks.
The first block is called Secure Network Encryption Protocol.
It provides data confidentiality, two-party data authentication and data freshness.
The second block is called $\mu$TESLA (for ``micro'' version of the Timed, Efficient, Streaming, Loss-tolerant Authentication Protocol) and assumes authenticated broadcast using one-way key chains constructed with secure hash functions.
Many other mechanisms related to authentication are resumed and compared in surveys such as~\cite{SOBMCN11}.




Denial of service is not the only type of attacks a WSN should resist to.
Security in general in sensor networks has attracted quite a lot of interest during the last years.
Hence it has been the subject of many studies in literature, as well as several state-of-the-art articles~\cite{DYK12,AD14}.

Confidentiality and integrity must be ensured to prevent attackers to access to, or to tamper sensitive data.
A number of solutions have been proposed~\cite{OX09}, many of them involving strong~\cite{SOBMCN11} and/or homomorphic~\cite{BBTY14} cryptography, some relying on other mechanisms such as multi-path based fragmentation of the packets~\cite{MMB13} or game theory~\cite{SWKC12}.

Authentication brings to participants the guaranty that the peer they are communicating with truly is what it pretends to be; that is another important point.
It has been deeply investigating as well~\cite{GWZC13}.
Many lightweight proposals for key management in WSNs have been suggested~\cite{GWZCK13,BSK13}.

Apart from those, there have been a variety of proposals to secure other elements, on a basis than any information about any aspect of the network might be valuable to an attacker.
Hence there are approaches, for instance, to secure the geographical location of the nodes through epidemical information dissemination~\cite{KDA14} as well as through more conventional mechanisms~\cite{GK13}.
