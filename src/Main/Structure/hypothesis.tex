% vim: set spelllang=fr foldmethod=marker:
\section{Hypothèses de travail}

\todo{À déplacer en \chapref{in} ou en début de \chapref{sa}? Terminologie entre sections 1 et 2 de ce chapitre?}

    \subsection{Terminologie employée}

        \paragraph{Capteurs et nœuds}
Un \rcsf est souvent représenté sous forme de graphe.
En conséquence, il est souvent fait références aux capteurs eux-mêmes sous le terme de « nœuds ».
Nous parlerons plutôt de capteurs lorsque nous évoquerons les « \rcs » eux-mêmes, ainsi que les mesures physiques qu'ils réalisent sur leur environnement, et plutôt de nœuds lorsque nous sommes amenés à travailler sur des graphes.
Mais la plupart du temps, ces deux termes peuvent être employés dans cette thèse de manière complètement interchangeable.
Leur alternance n'a le plus souvent que pour but de limiter les répétitions au sein d'une phrase.

        \paragraph{Nœuds normaux}
Les « nœuds normaux » ont deux significations possibles selon le contexte.
\begin{enumerate}
    \item Lorsqu'il est question de partition, de « clusterisation » du réseau, un nœud \textit{normal} est un nœud qui n'a pas été sélectionné pour assurer la fonction de « chef » du cluster, de \ch. De même lorsque des nœuds de contrôle (\cns ou \vns) ont été sélectionné, les nœuds \textit{normaux} sont les nœuds qui n'ont pas été sélectionnés (ni comme nœud de contrôle, ni comme \ch): ils poursuivent donc simplement leur rôle de collecte et de transmission des données mesurées.
    \item Lorsqu'il est question de \secu et d'attaques menées depuis l'intérieur du réseau, un nœud \textit{normal} désigne un nœud non compromis. Les termes « nœud légitime », « nœud sain » ont le même sens dans cette thèse. Ils sont à opposer au nœud « attaquant », « compromis », « corrompu » ou occasionnellement « cupide ».
\end{enumerate}

        \paragraph{Attaquant}
Le terme d'« attaquant » fera le plus souvent référence à l'entité consciente qui se trouve à l'origine d'une attaque, qu'il s'agisse d'une personne seule ou d'une organisation.
Il arrive toutefois que par abus de langage, l'\textit{attaquant} soit utilisé comme un raccourci pour « le nœud attaquant », c'est à dire le nœud corrompu par l'\textit{attaquant} réel, depuis lequel l'attaque est menée au sein du réseau.

        \paragraph{Exploitant}
L'\textit{exploitant} du réseau désignera l'entité (entité humaine ou organisation) qui exploite le \rcsfs: l'entité qui interagit avec la \sdb, de l'« autre côté » du réseau, pour récupérer et analyser les données collectées par les capteurs et remontées jusqu'à la \BS.
Sauf précision contraire, on fera généralement l'amalgame avec l'\textit{administrateur} du réseau, qui procède directement à sa mise en place et éventuellement à son maintien en fonction.

        \paragraph{Des capteurs conscients!}
Il va de soi qu'un capteur est un appareil dépourvu de vie et de conscience.
Par abus de langage, des raccourcis sont parfois empruntés dans cette thèse et leur prêtent des intentions ainsi qu'une durée de vie.
Ainsi il arrive que les capteurs « cherchent à », « essayent de », « veulent » réaliser des actions; ou bien qu'ils « prétendent » être de meilleurs candidats au poste de \ch.
De même, ils sont susceptibles de « mourir » lorsque leur batterie est épuisée.
Ces tournures, bien qu'abusives en termes d'exactitude du langage, permettent en général de simplifier les explications.
