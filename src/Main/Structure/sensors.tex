% vim: set spelllang=fr foldmethod=marker:
\section{Réseaux de capteurs sans fil}
\label{st:sec:contexte}

    \subsection{De quoi s'agit-il?}
    Les \rcsfs, ou \textit{\WSN} (pour \wsns en anglais), sont des réseaux constitués de petits appareils, les capteurs, ainsi que d'une \sdb.
\nomenclature{WSN}{\wsns}
Les capteurs échangent par communications hertziennes, en utilisant des protocoles tels ceux définis dans la pile \ieeee.
Le routage des paquets dans le réseau peut faire appel à l'un des nombreux protocoles développés à cet effet (par exemple: \aodv, \olsr), qu'il repose sur un algorithme centralisé (dirigé par une seule entité) ou distribué (exécuté par chaque entité du réseau).
Ils collectent des informations sur leur environnement et les font remonter à la \sdb.
Cette \sdb, ou \BS (pour \bs), parfois appelée \textit{puits} (ou \textit{sink} en anglais), est chargée de récolter et traiter les données provenant des capteurs.
Une fois les capteurs déployés, l'administrateur n'interagit plus avec le réseau que par le biais de la \sdb.
\nomenclature{BS}{\bs}

Quel que soit le protocole de liaison de données utilisé, il est rare que tous les capteurs d'un \WSN soit directement connectés les uns aux autres.
À la topologie d'un réseau donné est donc très souvent associé le graphe de connectivité du réseau.
Pour cette raison, dans la littérature comme dans la suite de ce document, il sera souvent fait référence aux capteurs sous le terme de \textit{nœuds} (\textit{nodes} en anglais).

    \subsection{Applications}
Le champ d'application des \WSN est très vaste.
Des réseaux peuvent être mis en place en forêt, par exemple, afin de détecter les départs de feu et de lutter plus efficacement contre les incendies.
D'autres peuvent être déployés en mer, afin de mesurer le taux de pollution des eaux.
En zones à risques, les capteurs peuvent être utilisés pour mesurer l'activité sismique ou volcanique du sol, et permettre une meilleure anticipation des phénomènes naturels.

Les \rcsfs sont également utilisés dans certains milieux critiques.
Un exemple serait leur usage dans le domaine du nucléaire.
Un autre, de taille, est leur exploitation intensive par les militaires sur les champs de batailles, afin de relier fantassins et véhicules de tous types; un maximum d'informations doit alors être remonté au centre de commandement, afin de permettre une supervision optimale des forces en mouvement.

Un autre domaine d'application en voie de développement est ce que l'on appelle l'\textit{\idx{Internet des objets}} (\textit{the Internet of things} en anglais), et qui consiste en quelque sorte à étendre \idx{Internet} au monde réel, par le biais d'une interconnexion réseau entre les objets de la vie courante.

    \subsection{Contraintes en ressources}
De par leur petite taille, et à cause de leurs déploiement dans des zones souvent difficiles d'accès, les capteurs n'embarquent qu'une quantité limitée de matériel, qui ne peut pas toujours être remplacé.
Les capteurs se retrouvent donc avec des capacités limitées, notamment en ce qui concerne:
\begin{itemize}
    \item les capacités de calcul: les processeurs embarqués sont relativement peu puissants.
        Les algorithmes exécutés par les capteurs doivent donc être de complexité relativement basse;
    \item les capacités de mémoire: les capteurs disposent de mémoire vive (RAM, \textit{Random Access Memory}, « mémoire à accès non séquentiel » en anglais) et d'un peu d'espace de stockage, mais ils ne sont pas du tout conçus pour sauvegarder de grandes bases de données.\nomenclature{RAM}{\textit{Random Access Memory}}
        Les informations récoltées doivent être acheminées à la \sdb, et non stockées sur le long terme par les capteurs eux-mêmes;
    \item l'énergie disponible: les capteurs disposent d'une batterie qui leur fournit une quantité d'énergie finie, et (la plupart du temps) non rechargeable.
        Il est donc essentiel de conserver à l'esprit une gestion parcimonieuse de l'énergie pour tout programme implémenté sur les capteurs.
        Des calculs importants, ainsi que des émissions/réceptions d'ondes électromagnétiques nombreuses ou mal gérées, sont les principaux facteurs d'un épuisement prématuré de la batterie.
\end{itemize}

Il est à noter qu'au regard de ces contraintes qui affectent les capteurs, la \sdb est considérée comme disposant de capacités « illimitées ».

    \subsection{Communications sans fil}

Comme l'indique leur nom, les réseaux de capteurs sans fil n'utilisent aucun câble physique pour communiquer entre eux ou avec la \sdb: toutes les transmissions sont effectuées par voie hertzienne.
Chaque capteur est équipé d'un module radio utilisé alternativement pour émettre et pour recevoir.
La plupart du temps ces modules sont capables de changer de fréquence de communication, ainsi que de moduler la puissance d'émission utilisée pour les transmissions.

Les protocoles de communication déployés sur cette architecture sont multiples (on parle d'\textit{encapsulation} des données).
Il faut pouvoir communiquer de pair à pair entre nœuds voisins, tout comme il faut être capable de faire parvenir un message à un nœud éloigner en faisant retransmettre les paquets par des nœuds relais successifs, d'assurer certains services comme le maintien de session, ou de gérer des applications.
Comme dans la plupart des réseaux informatiques, l'empilement des protocoles reprend donc le modèle \tcpip (schéma concret lui-même issu du modèle théorique \idx{OSI}, de l'anglais \textit{Open Systems Interconnection}), présenté en \figref{st:fig:tcpip}.
\nomenclature{OSI}{\textit{Open Systems Interconnection}}
\begin{figure}[!ht]
    \centering
    \begin{tabular}{c |c| l}
        \multicolumn{2}{c}{} & Exemples:\\
        \cline{2-2}
        5           & Application & HTTP, FTP, SSH\\
        \cline{2-2}
        4           & Transport   & TCP, UDP\\
        \cline{2-2}
        3           & Réseau      & IP\\
        \cline{2-2}
        2           & Liaison     & \ieeee (\csmaca)\\
        \cline{2-2}
        1           & Physique    & ondes électromagnétiques\\
        \cline{2-2}
     \end{tabular}
    \medskip
    \caption{Modèle TCP/IP}\label{st:fig:tcpip}
\end{figure}
\nomenclature{TCP}{\textit{Transmission Control Protocol}}
\nomenclature{IP}{\textit{Internet Protocol}}
\nomenclature{UDP}{\textit{User Datagram Protocol}}
\nomenclature{HTTP}{\textit{Hypertext Transfer Protocol}}
\nomenclature{FTP}{\textit{File Transfer Protocol}}
\nomenclature{SSH}{\textit{Secure Shell}}

Certains standards normalisés définissent l'usage de protocoles spécifiques sur les trois premières couches.
Les principaux standards qui sont employés dans les réseaux de capteurs sont les piles \ieeee (correspondant à la marque \wifi) et \ieeeff (sur laquelle sont basée la marque \zigbee et le standard \ietf \slowpan), et dans une moindre mesure la pile \ieeefo (correspondant à la marque \bluetooth).
\nomenclature{IETF}{\textit{Internet Engineering Task Force}}
\nomenclature{IEEE}{\textit{Institute of Electrical and Electronics Engineers}}
\nomenclature{6LoWPAN}{\textit{IPv6 over Low power Wireless Personal Area Networks}}

            \paragraph{La couche physique}
Cette couche concerne l'émission et la réception en soi des ondes électromagnétiques, et l'encodage utilisé pour leur faire porter des valeurs numériques (par opposition à un signal analogique).
Nous n'aurons pas besoin de l'étudier en détail ici.

\paragraph{La couche de liaison de données}\label{st:ssec:mac}
La deuxième couche du modèle fournit les moyens fonctionnels et procéduraux pour le transfert des données entre deux entités du réseau.
Elle permet aussi, le plus souvent, de détecter et éventuellement corriger certaines erreurs survenues sur la couche physique (en cas de perturbation ou dégradation du signal électromagnétique).
Elle se décompose en deux sous-couches: la couche de contrôle de la liaison logique (LLC, pour \textit{Logical Link Control} en anglais, « contrôle de la liaison logique ») et la couche du contrôle d'accès au support (MAC, pour \textit{Media Access Control}, « contrôle d'accès au support »).
LLC est la sous-couche haute, utilisée pour fiabiliser la sous-couche MAC, intervient peu dans les réseaux de capteurs.
Le protocole de couche MAC définit la manière dont les différents agents du réseau accèdent au médium de transmission de façon à limiter les collisions, et à garantir un accès le plus souvent équivalent au médium pour tous les nœuds.
Les différents modes d'accès au médium existants sont résumés dans la \tabref{st:tab:mac}; certains consistent à créer des canaux de transmission distincts, tandis que d'autres déterminent l'accès à une même bande de fréquence en instaurant des règles.
\nomenclature{MAC}{\textit{Media Access Control}}
\nomenclature{LLC}{\textit{Logical Link Control}}

\begin{table}[!ht]
    \caption{Méthodes d'accès au médium de transmission}\label{st:tab:mac}
    \centering
    \medskip
    \begin{small}
        \begin{tabular}{m{.2\textwidth}|m{.2\textwidth}|m{.48\textwidth}}
            \toprule
            \textsc{Nom anglais} & \textsc{Traduction} & \textsc{Description}\\
            \midrule
            \multicolumn{3}{c}{Commutation de circuits et création de canaux}\\
            \midrule
            \textit{Frequency Division Multiple Access} (\fdma) & Accès multiple par répartition en fréquence & Plusieurs canaux basés sur des fréquences différentes\\
            \midrule
            \textit{Code division multiple access} (\cdma) & Accès multiple par répartition en code & Étalement du spectre de fréquence utilisé en conjonction avec techniques comme les sauts de fréquences ou la génération de bruit pseudo-aléatoires (avec la même séquence pseudo-aléatoire appliquée au signal côté émetteur comme côté destinataire)\\
            \midrule
            \textit{Time division multiple access} (\tdma) & Accès multiple à répartition dans le temps & Un seul canal dont l'accès est réparti par créneaux dans le temps\\
            \midrule
            \textit{Space division multiple access} (\sdma) & Accès multiple à répartition dans l'espace & Plusieurs canaux spatiaux obtenus à l'aide d'antennes directionnelles. À noter: les antennes directionnelles augmentent sensiblement le cout de production des capteurs.\\
            \midrule
            \multicolumn{3}{c}{Mode d'accès par paquet}\\
            \midrule
            \textit{Contention based random multiple access methods} & Accès par contention & Contention par le nœud du paquet à envoyer jusqu'à ce que le protocole le lui autorise. Dans cette catégorie se trouve notamment le protocole \csmaca (\textit{Carrier Sense Multiple Access with Collision Avoidance}, accès multiple par écoute du canal avec esquive de collision), très utilisé dans les réseaux sans fil (\ieeee entre autres).\\
            \midrule
            \textit{Resource reservation (scheduled) packet-mode protocols} & Réservation des ressources & Réservation par un nœud des ressources (par exemple: créneau temporel) nécessaires à la transmission)\\
            \midrule
            \multicolumn{3}{p{.95\textwidth}}{D'autres modes d'accès par paquet (\textit{token passing, polling}) existent mais ne sont pas utilisés dans les réseaux de capteurs}\\
            \bottomrule
         \end{tabular}
     \end{small}
\end{table}
\nomenclature{FDMA}{\textit{Frequency Division Multiple Access}}
\nomenclature{CDMA}{\textit{Code Division Multiple Access}}
\nomenclature{SDMA}{\textit{Space Division Multiple Access}}
\nomenclature{TDMA}{\textit{Time  Division Multiple Access}}
\nomenclature{CSMA}{\textit{Carrier Sense Multiple Access}}
\nomenclature{CSMA/CA}{\textit{Carrier Sense Multiple Access with Collision Avoidance}}

De ces modes d'accès sont dérivés de très nombreux protocoles de couche MAC, dont plusieurs ont été conçus spécifiquement pour les réseaux de capteurs~\cite{YB09}.
Par exemple, en dehors des standards \ieee, le protocole \smac~\cite{YHE02} fait alterner périodes « actives » et périodes de sommeil aux capteurs afin de préserver leur batterie.
Les capteurs sont associés dans des groupes (qui ne correspondent pas tout à fait à des clusters) dont tous les membres sont en éveil de manière simultanée, afin de pouvoir communiquer à l'aide d'une version modifiée de \csmaca tel que défini dans le standard \ieeee.
\nomenclature{S-MAC}{\textit{Sensor-MAC}}

            \paragraph{La couche réseau}
Cette couche, pour les capteurs sans fil, repose le plus souvent sur le protocole \ip (\textit{Internet Protocol} an anglais), que ce soit en version 4 ou 6 (la version 6 est notamment utilisée avec la pile \slowpan), pour l'adressage, et sur un protocole de routage qui détermine comment les paquets sont retransmis saut après saut dans le réseau.
Des exemples classiques de ces algorithmes de routage incluent \aodv (\textit{Ad hoc On-Demand Distance Vector Routing})~\cite{aodv} ou son évolution \dsr~\cite{dsr} (\textit{Dynamic Source Routing}), \olsr~\cite{olsr} (\textit{Optimized Link State Routing Protocol}) ou ses évolutions \dsdv (\textit{Destination-Sequenced Distance Vector routing}) et \batman (\textit{Better Approach To Mobile Adhoc Networking}), \etc.
\nomenclature{AODV}{\textit{Ad hoc On-Demand Distance Vector Routing}}
\nomenclature{DSR}{\textit{Dynamic Source Routing}}
\nomenclature{OLSR}{\textit{Optimized Link State Routing Protocol}}
\nomenclature{DSDV}{\textit{Destination-Sequenced Distance Vector routing}}
\nomenclature{B.A.T.M.A.N.}{\textit{Better Approach To Mobile Adhoc Networking}}

            \paragraph{La couche transport}
Cette couche gère les communications de bout en bout entre processus.
Les protocoles les plus fréquemment utilisés, \tcp (\textit{Transmission Control Protocol}) et \udp (\textit{User Datagram Protocol}) assurent l'ordonnancement des paquets et permettent d'éviter les pertes, les doublons et la corruption des paquets.
Le protocole \tcp permet en plus à deux entités de communiquer en mode connecté en établissant une session (début, fin et validation des échanges).
En raison des données de contrôle que ces protocoles impliquent, ils ne sont pas systématiquement implémentés dans les réseaux de capteurs (ils le sont parfois dans des versions allégées).

            \paragraph{La couche application}
La dernière couche gère les échanges applicatifs entre les différents agents du réseau, et encapsule les données utiles au format désiré pour leur traitement final.
Il s'agit de la couche la plus haut niveau  du modèle.
Elle assure un service propre à l'application déployée, et les protocoles déployés à ce niveau dépendent donc totalement de l'objectif final du réseau, il n'y a donc pas de standard propre aux \rcs.
