% vim: set spelllang=fr foldmethod=marker ft=tex:
\vfil
\lettrineh{L}{es \rcs}, leur partition en clusters, et les différentes solutions existantes pour lutter contre le \dds forment une charpente sur laquelle il est possible de construire encore.
Les chapitres qui viennent, à commencer par celui-ci, présentent les différentes contributions apportées au cours de la thèse.
Ces contributions sont principalement centrées sur le renouvellement et l'établissement d'un processus efficace de sélection pour les nœuds de surveillance d'un \rcs.

La première d'entre elles détaille l'architecture ainsi que les mécanismes employés, met en place le renouvellement de la sélection, et propose un processus aléatoire pour la désignation des nœuds de surveillance.
Elle est évaluée par des simulations, puis à l'aide d'outils de modélisation.
Concrètement:
\begin{itemize}
    \item dans un premier temps est présenté un jeu de résultats numériques obtenu par des simulations réalisées avec le logiciel \nsii sur une grille de capteurs, avec et sans renouvellement des nœuds de surveillance, afin d'éprouver la méthode proposée;
    \item le système est ensuite l'objet d'une première représentation, sous forme de chaines de \textsc{Markov}. Des mesures analytiques sont obtenues à partir des états stationnaires, pour obtenir la probabilité de la détection des attaques dans le cadre de ce modèle;
    \item une seconde modélisation du mécanisme de détection est exprimée à l'aide de réseaux de \textsc{Petri} stochastiques généralisés, et accompagnée de propriétés établies à l'aide de logique stochastique avec automates hybrides.
\end{itemize}
\vfil
