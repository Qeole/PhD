% vim: set spelllang=fr foldmethod=marker:
\vfil
\lettrineh{L}{es \rcs}, leur partition en clusters, et les différentes solutions existantes pour lutter contre le \dds forment une charpente sur laquelle il est possible de construire encore.
Les chapitres qui viennent, à commencer par celui-ci, présentent les différentes contributions apportées au cours de la thèse.
Ces contributions sont principalement centrées sur le renouvellement et l'établissement d'un processus efficace de sélection pour les nœuds de surveillance d'un \rcs.
La première d'entre elles détaille l'architecture ainsi que les mécanismes employés, met en place le renouvellement de la sélection, et propose un processus aléatoire pour la désignation des nœuds de surveillance.
Le système obtenu est également modelé à l'aide de différents outils: chaines de \textsc{Markov} à temps continu, réseaux de \textsc{Petri} stochastiques généralisés étendus, et logique stochastique avec automates hybrides.
Des résultats numériques, issus de simulations réalisées avec le logiciel \nsii, viennent ensuite éprouver la solution proposée.
\vfil

\section{Contexte}

\subsection{Hypothèses de travail}
