Composés d'appareils fortement limités en ressources (puissance de calcul, mémoire et énergie disponible) et qui communiquent par voie hertzienne, les réseaux de capteurs sans fil composent avec leurs faibles capacités pour déployer une architecture de communication de manière autonome, collecter des données sur leur environnement et les faire remonter jusqu'à l'utilisateur.
Des «transports intelligents» à la surveillance du taux de pollution environnemental, en passant par la détection d'incendies ou encore l'«Internet des objets», ces réseaux sont aujourd'hui utilisés dans une multitude d'applications.
Certaines d'entre elles, de nature médicale ou militaire par exemple, ont de fortes exigences en matière de sécurité.

Les travaux de cette thèse se concentrent sur la protection contre les attaques dites par «déni de service», qui visent à perturber le fonctionnement normal du réseau.
Ils sont basés sur l'utilisation de capteurs de surveillance, qui sont périodiquement renouvelés pour répartir la consommation en énergie.
De nouveaux mécanismes sont introduits pour établir un processus de sélection efficace de ces capteurs, en optimisant la simplicité de déploiement (sélection aléatoire), la répartition de la charge énergétique (sélection selon l'énergie résiduelle) ou encore la sécurité du réseau (élection démocratique basée sur un score de réputation).
Sont également fournis différents outils pour modéliser les systèmes obtenus sous forme de chaines de Markov à temps continu, de réseaux de Petri stochastiques (réutilisables pour des opérations de model checking) ou encore de jeux quantitatifs.
