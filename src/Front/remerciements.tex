% vim: set spelllang=fr:
\chapter{Remerciements}

Décrire le doctorat comme la seule préparation d'un diplôme serait une injustice criante.
Il y aurait beaucoup à en dire, mais j'ai choisi, avant de plonger dans la rigueur des sciences informatiques, d'en retenir un point de vue spécifique: le doctorat est un ensemble de rencontres.

Il y a d'abord les rencontres en amont.
Lynda m'a eu comme élève en Master, et je lui dois beaucoup de cette thèse: sujet, financement, encadrement, débats et coopération.
Du début à la fin, elle m'a guidé et surtout m'a fait confiance, et je l'en remercie profondément.

Pendant la thèse, il y a ceux que j'ai rencontrés en arrivant, parce qu'ils étaient au LACL avant moi ou bien parce qu'ils sont arrivés tout en même temps.
Les doctorants sont sans doute ceux avec qui j'ai le plus échangé: Dimitrios, Louis, Mohamed, Muath m'ont précédé, Yoann m'a suivi de bureau en bureau, de théorie fumeuse en complot pour refaire le monde, machine après machine.
C'est tout naturellement que je les remercie, de même que je remercie l'ensemble de l'équipe du laboratoire, pour leur accueil et leurs nombreux conseils.

Et il y a toutes les rencontres effectuées pendant, au cours de, durant cette thèse; pour travailler, pour enseigner, par chance, par hasard.
La recherche m'a permis de travailler avec plusieurs personnes: merci à Paolo et à Mathieu, pour ne citer qu'eux.
À Créteil comme à Dauphine, j'ai découvert et pris goût à l'enseignement; j'ai certes croisé beaucoup d'élèves, mais aussi rencontré des collègues avec qui l'organisation des cours est un plaisir.
Merci notamment à Antoine, Julien, Luidnel et Pascal pour les cours de systèmes d'exploitation.
Et pour le Beaufort.

Toujours pendant la thèse, et au fil des ans, d'autres individus, plus ou moins recommandables, ont rejoint les rangs des doctorants du LACL; à défaut d'avoir réussi à leur imposer la supériorité de ma condition d'« ancien », je m'en suis fait des amis.
Merci à Aurélien, Martin, Nghi, Raouf, Rodica, Sergiu, Thomas et Victor pour les repas et les discussions, les logiciels libres, les dessins étranges, les métalangages et les comorphismes.
Entre autres choses.

\vfil
{\it
This Ph.~D. brought me to Liverpool for two months, and I was lucky enough to meet wonderful people there.
I have to thank Prof \textsc{Merabti} and his team for their hospitality at LJMU.
Also I am grateful to Aíne, Brett, David, Mark, Rob, Will and the many other Ph.~D. students there from whom I received a more-than-warm welcome.
}
\vfil

Il y les rencontres futures, qui je l'espère seront nombreuses, et sauront profiter à tous.
Mais il m'est difficile de leur dédier des remerciements aujourd'hui…

Enfin il est de ces rencontres qui sont hors contexte, intemporelles, qui ne sont ni à l'instant ni au lieu de la thèse.
Ma famille et mes proches, qui m'ont soutenu, supporté parfois pendant ces années, se retrouvent lésés: car je ne pourrai jamais les remercier assez.
