Wireless sensor networks are made of small devices with low resources (low computing power, little memory and little energy available), communicating through electromagnetic transmissions.
In spite of these limitations, sensors are able to self-deploy and to auto-organize into a network collecting, gathering and forwarding data about their environment to the user.
Today those networks are used for many purposes: “intelligent transportation”, monitoring pollution level in the environment, detecting fires, or the “Internet of things” are some example applications involving sensors.
Some of them, such as applications from medical or military domains, have strong security requirements.

The work of this thesis focuses on protection against “denial of service” attacks which are meant to harm the good functioning of the network.
It relies on the use of monitoring sensors: these sentinels are periodically renewed so as to better balance the energy consumption.
New mechanisms are introduced so as to establish an efficient selection process for those sensors: the first one favors the ease of deployment (random selection), while the second one promotes load balancing (selection based on residual energy) and the last one is about better security (democratic election based on reputation scores).
Furthermore, some tools are provided to model the system as continuous-time Markov chains, as stochastic Petri networks (which are reusable for model checking operations) or even as quantitative games.
